% ⓒ 2017 Monika Sturm, Daniel Borchmann
% This work is licensed under the Creative Commons Attribution-ShareAlike 4.0
% International License. To view a copy of this license, visit
% http://creativecommons.org/licenses/by-sa/4.0/.

\documentclass[german]{latteachCD}[2017/03/28]

\usepackage[sheetnumber=3]{theolog}

% Entscheidbarkeit, Unentscheidbarkeit, Reduktionen

\begin{document}

\maketitle

\enlargethispage{2\baselineskip}

\begin{mdframed}
  Die folgenden Aufgaben werden nicht in den Übungen besprochen und dienen der
  Selbstkontrolle.

  \renewcommand{\theexercise}{\Alph{exercise}}
  \setcounter{exercise}{4}

  \begin{exercise}
    Geben Sie eine Turing-Maschine $\mathcal{A}_{\mathsf{mod2}}$ an, die die
    Funktion $f\colon \mathbb N \to \mathbb N$ mit $f(x) = (x \bmod 2)$
    berechnet.  Stellen Sie dabei die Zahlen in unärer Kodierung dar.
  \end{exercise}

  \begin{exercise}
    Es sei $f \colon \mathbb N \to \mathbb N$ mit $f(x) = \lfloor
    \log_{10}(x)\rfloor$.  Geben Sie ein WHILE-Programm an, welches $f$
    berechnet.
  \end{exercise}
  \vspace*{-0.5\baselineskip}
\end{mdframed}

\vspace*{0.5\baselineskip}

\setcounter{exercise}{0}

\begin{exercise}
  % Quelle: Sipser, Aufgab 5.5
  Zeigen Sie, dass es keine Many-One-Reduktion vom Halteproblem
  $\Slang{P}_{\mathsf{halt}}$ von Turing-Maschinen auf des Leerheitsproblem
  \begin{equation*}
    \Slang{P}_{\mathsf{leer}} \coloneqq \{ \mathsf{enc}(\Smach{M}) \mid
    \mathcal{L}(\Smach{M}) = \emptyset \}
  \end{equation*}
  von Turing-Maschinen gibt.
\end{exercise}

\begin{exercise}
  % Quelle unbekannt
  Es sei
  \begin{equation*}
    T \coloneqq \{ \mathsf{enc}(\Smach{M}) \mid \Smach{M} \text{ ist eine Turing-Maschine,
      welche $w^{\mathcal{R}}$ akzeptiert, falls sie $w$ akzeptiert}\},
  \end{equation*}
  wobei $w^{\mathcal{R}}$ das zu $w$ umgekehrte Wort ist.  Zeigen Sie, dass $T$
  nicht entscheidbar ist.
\end{exercise}

\begin{exercise}
  % Quelle: Sipser, Aufgabe 4.14
  Es sei
  \begin{equation*}
    L \coloneqq \{\mathsf{enc}(G)\#\#\mathsf{enc}(x) \mid G \text{ kontextfreie
      Grammatik und $x$ Teilwort eines Wortes aus } L(G)\},
  \end{equation*}
  wobei $\mathsf{enc}(G)$ eine Kodierung von $G$ ist.  Zeigen Sie, dass $L$ auf
  das Komplement des Leerheitsproblems kontextfreier Grammatiken
  many-one-reduziert werden kann.

  \Hinweis{Nutzen Sie die Tatsache, dass der Schnitt einer regulären und einer
    kontextfreien Sprache wieder kontextfrei ist.}
\end{exercise}

\begin{exercise}
  % Quelle: Sipser, Aufgabe 5.10
  Zeigen Sie, dass jede semi-entscheidbare Sprache $L$ auf das Halteproblem
  $\Slang{P}_{\mathsf{halt}}$ many-one-reduziert werden kann.
\end{exercise}

\end{document}
