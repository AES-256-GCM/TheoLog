% ⓒ 2017 Monika Sturm, Daniel Borchmann
% This work is licensed under the Creative Commons Attribution-ShareAlike 4.0
% International License. To view a copy of this license, visit
% http://creativecommons.org/licenses/by-sa/4.0/.

\documentclass[german]{latteachCD}[2017/03/28]

\usepackage[sheetnumber=4]{theolog}

\begin{document}

\maketitle

\begin{mdframed}
  Die folgenden Aufgaben werden nicht in den Übungen besprochen und dienen der
  Selbstkontrolle.

  \renewcommand{\theexercise}{\Alph{exercise}}
  \setcounter{exercise}{7}

  \begin{exercise}
    % Markus Krötzsch
    Sei $\Slang{L}$ eine unentscheidbare Sprache. Zeigen Sie:
    \begin{enumerate}
    \item\label{item:1} $\Slang{L}$ hat eine Teilmenge $\Slang{T}\subseteq \Slang{L}$, die
      entscheidbar ist.
    \item\label{item:2} $\Slang{L}$ hat eine Obermenge $\Slang{O}\supseteq
      \Slang{L}$, die entscheidbar ist.
    \item Es gibt jeweils nicht nur eine sondern unendlich viele entscheidbare
      Teilmengen bzw. Obermengen wie in~(\ref{item:1}) und~(\ref{item:2}).
    \end{enumerate}
  \end{exercise}

  \begin{exercise}
    \dots
  \end{exercise}

\end{mdframed}

\vspace*{0.5\baselineskip}

\setcounter{exercise}{0}

\dots

\end{document}
