% ⓒ 2017 Monika Sturm, Daniel Borchmann
% This work is licensed under the Creative Commons Attribution-ShareAlike 4.0
% International License. To view a copy of this license, visit
% http://creativecommons.org/licenses/by-sa/4.0/.

\documentclass[german]{latteachCD}[2017/03/28]

% Einführung in die Komplexitätstheorie

\usepackage[sheetnumber=5]{theolog}

\begin{document}

\maketitle

\begin{mdframed}
  Die folgenden Aufgaben werden nicht in den Übungen besprochen und dienen der
  Selbstkontrolle.

  \renewcommand{\theexercise}{\Alph{exercise}}
  \setcounter{exercise}{7}

  \begin{exercise}
    % Markus Krötzsch
    Sei $\Slang{L}$ eine unentscheidbare Sprache. Zeigen Sie:
    \begin{enumerate}
    \item\label{item:1} $\Slang{L}$ hat eine Teilmenge $\Slang{T}\subseteq \Slang{L}$, die
      entscheidbar ist.
    \item\label{item:2} $\Slang{L}$ hat eine Obermenge $\Slang{O}\supseteq
      \Slang{L}$, die entscheidbar ist.
    \item Es gibt jeweils nicht nur eine sondern unendlich viele entscheidbare
      Teilmengen bzw. Obermengen wie in~(\ref{item:1}) und~(\ref{item:2}).
    \end{enumerate}
  \end{exercise}

  \renewcommand{\theexercise}{\BoldGreek{exercise}}
  \setcounter{exercise}{0}

  \begin{exercise}
    % TheoLog 2014
    \vspace*{-\baselineskip}
    \begin{enumerate}
    \item Beschreiben Sie mit eigenen Worten die Probleme, die in
      $\PTime$ liegen.
    \item Beschreiben Sie mit eigenen Worten die Probleme, die in
      $\NP$ liegen.
    \item Beschreiben Sie mit eigenen Worten die Probleme, die in $\PSpace$
      liegen.
    \item Erläutern Sie, warum $\PTime \subseteq \NP \subseteq \PSpace$ gilt.
    \item Beschreiben Sie für $\mathcal{C} = \NP$ und $\mathcal{C} = \PSpace$,
      wann ein Problem \enquote{$\mathcal{C}$-hart} bzw.\@
      \enquote{$\mathcal{C}$-vollständig} ist.
    \end{enumerate}
  \end{exercise}

\end{mdframed}

\vspace*{0.5\baselineskip}

\setcounter{exercise}{0}

\begin{exercise}
  % TheoLog 2014
  Welche der folgenden Aussagen sind wahr?  Begründen Sie Ihre Antwort.
  \begin{enumerate}
  \item Falls $\PTime \neq \NP$ gilt, dann auch $\PTime \cap \NP \neq \emptyset$.
  \item Es gibt Probleme, die $\NP$-hart, aber nicht $\NP$-vollständig sind.
  \item Polynomielle Reduzierbarkeit ist nicht transitiv.
  \item Ist $\Slang L_{2} \in \PTime$ und $\Slang L_{1} \leq_{p} \Slang L_{2}$,
    dann ist auch $\Slang L_{1} \in \PTime$.
  \item Ist $\Slang L_{1}$ eine $\NP$-vollständige Sprache und gilt $\Slang
    L_{1} \leq_{p} \Slang L_{2}$, dann ist auch $\Slang L_{2}$ eine
    \NP-vollständige Sprache.
  \item Ist $\Slang L_{2}$ eine $\NP$-vollständige Sprache und gilt $\Slang
    L_{1} \leq_{p} \Slang L_{2}$, dann ist auch $\Slang L_{1}$ eine
    \NP-vollständige Sprache.
  \end{enumerate}
\end{exercise}

\begin{exercise}
  % Sipser
  Zeigen Sie, dass das Wortproblem deterministischer endlicher Automaten in
  \LogSpace liegt: ist
  \begin{equation*}
    \Slang P_{\mathsf{DFA}} \coloneqq \{\,\enc(\Smach A)\#\#\enc(w) \mid
    \Smach A \text{ ist ein DFA, der $w$ akzeptiert}\,\},
  \end{equation*}
  dann gilt $\Slang P_{\mathsf{DFA}} \in \LogSpace$.
\end{exercise}

\begin{exercise}
  % TheoLog 2014
  Es sei $\Slang{L} \coloneqq \{\,a^{n} \mid n \in \mathbb N \text{ ist keine
    Primzahl}\,\}$.  Zeigen Sie, dass $\Slang L \in \NP$ gilt.
\end{exercise}

\begin{exercise}
  % Complexity Theory Lecture by Markus Krötzsch and Daniel Borchmann
  Zeigen Sie: ist $\PTime = \NP$, dann gibt es einen Algorithmus, der in
  polynomieller Zeit für jede erfüllbare aussagenlogische Formel eine erfüllende
  Belegung findet.
\end{exercise}

\end{document}
