% ⓒ 2017 Monika Sturm, Daniel Borchmann
% This work is licensed under the Creative Commons Attribution-ShareAlike 4.0
% International License. To view a copy of this license, visit
% http://creativecommons.org/licenses/by-sa/4.0/.

\documentclass[german]{latteachCD}[2017/03/28]

% NP-Vollständigkeit

\usepackage[sheetnumber=6]{theolog}
\title{Musterlösung zu Übungsblatt~6}

\usepackage{ragged2e}

\begin{document}

\maketitle

\vspace*{0.5\baselineskip}
\setcounter{exercise}{2}
\justifying

\begin{exercise}
  % Folklore
  Sei $\Sigma$ ein Alphabet und $A, B \subseteq \Sigma^{*}$.  Wir sagen, dass
  $A$ auf $B$ in \emph{logarithmischen Platz reduzierbar ist}, und schreiben $A
  \leq_{\ell} B$, falls es eine Many-One-Reduktion von $A$ nach $B$ gibt, die in
  logarithmischen Platz berechenbar ist.  Zeigen Sie: gilt $A \leq_{\ell} B$ und
  $B \leq_{\ell} C$, dann gilt auch $A \leq_{\ell} C$.
\end{exercise}

\emph{Lösung}\/: Wir zeigen, dass für zwei logspace-berechenbare Funktion $f,
g\colon \Sigma^{*} \to \Sigma^{*}$ auch $f \circ g$ logspace-berechenbar ist.
Seien dazu $\Smach M_{f}$ und $\Smach M_{g}$ Turing-Maschinen, die mit
logarithmischer Platzbeschränkung die Funktionen $f$ und $g$ berechnen.

Eine erste Idee, eine Turing-Maschine $\Smach M$ zu erhalten, die $f \circ g$
berechnet, ist, zuerst $\Smach M_{g}$ auf der Eingabe aufzurufen, das
Zwischenergebnis zu speichern, und dann $\Smach M_{f}$ auf diesem
Zwischenergebnis laufen zu lassen.  Dieser Idee funktioniert jedoch nicht: zwar
benutzt $\Smach M_{g}$ bei Eingabe $w \in \Sigma^{*}$ nur zusätzlich
logarithmischen Platz zur Berechnung von $g(w)$.  Dieses Ergebnis kann jedoch
polynomiell groß sein in der Länge von $w$ -- und damit exponentiell in der
Größe des zur Verfügung stehenden Platzes, der ja logarithmisch in der Größe von
$w$ beschränkt ist.  Damit kann das Zwischenergebnis $g(w)$ nicht vollständig
gespeichert werden und dieser Ansatz funktioniert nicht.

Wir können aber diese Idee so modifizieren, dass sie funktioniert!  Dazu
berechnen wir die Zeichen von $g(w)$ \enquote{on demand}: wir verändern $\Smach
M_{g}$ so, dass sie nur das $k$-te Symbol von $g(w)$ berechnet.  Dies kann
erreicht werden, indem $\Smach M_{g}$ mit einem weiteren Zähler $p$ versehen
wird, der um eins hochgezählt wird, wann immer $\Smach M_{g}$ ein Symbol
ausgeben möchte.  Ist der Wert von $p$ gleich $k$, gibt $\Smach M_{g}$ das
entsprechende Zeichen aus und hält an.

Um $f(g(w))$ in logspace zu berechnen, gehen wir nun wie folgt vor: wir
simulieren die Berechnung von $\Smach M_{f}$.  Wann immer diese Berechnung ein
Symbol von $g(w)$ lesen möchte, simulieren wir $\Smach M_{g}$ wie oben
beschrieben.  Beide Simulationen können in logspace durchgeführt werden, und
damit kann auch $f(g(w))$ in logspace berechnet werden.

Diese Berechnung von $f(g(w))$ ist recht ineffizient, da Symbole von $g(w)$
möglicherweise mehrfach berechnet werden.  Wir haben aber potentiell nicht genug
Platz, um das gesamte Wort $g(w)$ zu speichern.  Wir tauschen also
\enquote{Platz gegen Zeit}.

\end{document}

%  LocalWords:  logspace
