% ⓒ 2017 Monika Sturm, Francesco Kriegel, Daniel Borchmann
% This work is licensed under the Creative Commons Attribution-ShareAlike 4.0
% International License. To view a copy of this license, visit
% http://creativecommons.org/licenses/by-sa/4.0/.

\documentclass[german]{latteachCD}[2017/03/28]

% Einführung Logik

\usepackage[sheetnumber=9]{theolog}

\begin{document}

\maketitle

\begin{mdframed}
  Die folgenden Aufgaben werden nicht in den Übungen besprochen und dienen der
  Selbstkontrolle.

  \renewcommand{\theexercise}{\Roman{exercise}}
  \setcounter{exercise}{0}

  \begin{exercise}
    % TheoLog 2014
    Geben Sie für die Formel
    \begin{equation*}
      F = \forall x. \exists y. (p(c_{1},z) \land (q(x,c_{2},z) \lor p(c_{2},y))),
    \end{equation*}
    wobei $c_{1}, c_{2}$ Konstanten sind, folgendes an:
    \begin{enumerate}
    \item die Menge aller Teilformeln;
    \item die Menge aller Terme;
    \item die Menge aller Variablen, mit Unterscheidung freier und gebundener
      Variablen;
    \item ein Interpretation $\mathcal{I}$ und eine Zuweisung $\mathcal{Z}$ für
      $\mathcal{I}$, so dass $\mathcal{I}, \mathcal{Z} \models F$.
    \end{enumerate}
  \end{exercise}

  \begin{exercise}
    % TheoLog 2014
    Zeigen Sie die folgenden Aussagen:
    \begin{enumerate}
    \item Es gilt $\{F\} \models G$ genau dann, wenn $F \to G$
      allgemeingültig ist.
    \item Es gilt $\{F_{1}, \dots, F_{k}\} \models G$ genau dann, wenn
      $\bigwedge_{i=1}^{k} F_{i} \to G$ allgemeingültig ist.
    \end{enumerate}
  \end{exercise}

  \begin{exercise}
    % TheoLog 2014
    Seien $F$, $G$ Formeln und $x$ eine Variable.  Zeigen Sie, dass dann gilt
    \begin{equation*}
      \exists x.(F \to G) \equiv \forall x. F \to \exists x. G.
    \end{equation*}
  \end{exercise}
  \vspace*{-1.7\baselineskip}
\end{mdframed}

\setcounter{exercise}{0}

\enlargethispage{2\baselineskip}

\begin{exercise}
  % TheoLog 2014
  Welche der angegebenen Strukturen sind Modelle der folgenden Formel?
  \begin{equation*}
    \forall x.\, p(x,x) \land \forall x, y.\, ((p(x,y) \land
      p(y,x)) \to x \approx y)
    \land \forall x, y, z.\, (p(x,y) \land p(y,z) \to p(x,z))
  \end{equation*}
  \begin{enumerate}
  \item $\mathcal{I}_{1}$ mit Grundmenge $\mathbb N$ und $p^{\mathcal{I}_{1}} =
    \{\,(m,n) \mid m < n \,\}$;
  \item $\mathcal{I}_{2}$ mit Grundmenge $\mathbb N$ und $p^{\mathcal{I}_{2}} =
    \{\,(n,n+1) \mid n \in \mathbb N \,\}$;
  \item $\mathcal{I}_{3}$ mit Grundmenge $\mathbb N$ und $p^{\mathcal{I}_{3}} =
    \{\,(m,n) \mid m \text{ teilt } n\,\}$;
  \item $\mathcal{I}_{4}$ mit Grundmenge $\Sigma^{*}$ für ein Alphabet $\Sigma$
    und $p^{\mathcal{I}_{4}} = \{\,(x,y) \mid x \text{ ist Präfix von } y\,\}$;
  \item $\mathcal{I}_{5}$ mit Grundmenge $\subsets{M}$ für eine Menge $M$ und
    $p^{\mathcal{I}_{5}} = \{\,(X,Y) \mid X \subseteq Y\,\}$.
  \end{enumerate}
\end{exercise}

\begin{exercise}
  % TheoLog 2014
  \begin{enumerate}
  \item Geben Sie je eine erfüllbare Formel in Prädikatenlogik mit Gleichheit
    an, so dass alle Modelle
    \begin{enumerate}
    \item höchstens drei,
    \item mindestens drei,
    \item genau drei
    \end{enumerate}
    Elemente in der Grundmenge besitzen.
  \item Geben Sie je eine erfüllbare Formel in Prädikatenlogik mit Gleichheit
    an, so dass das zweistellige Relationensymbol $p$ in jedem Modell als der
    Graph einer
    \begin{enumerate}
    \item injektiven Funktion,
    \item surjektiven Funktion,
    \item bijektiven Funktion
    \end{enumerate}
    interpretiert wird.

    (Der Graph einer Funktion $f \colon A \to B$ ist die Relation $\{\,(x,y) \in
    A \times B \mid f(x) = y\,\}$.)
  \end{enumerate}
\end{exercise}

\begin{exercise}
  % TheoLog 2014
  Welche der folgenden Aussagen sind wahr?  Begründen Sie Ihre Antwort.
  \begin{enumerate}
  \item Sind $\Gamma$ und $\Gamma'$ Menge von prädikatenlogischen Formeln, dann
    folgt aus $\Gamma \subseteq \Gamma'$ und $\Gamma \models F$ auch $\Gamma'
    \models F$.
  \item Jede aussagenlogische Formel ist eine prädikatenlogische Formel.
  \item Eine prädikatenlogische Formel $F$ ist genau dann allgemeingültig,
    wenn $\lnot F$ unerfüllbar ist.
  \item Es gilt
    \begin{equation*}
      \{\,\forall x, y.\, (p(x,y) \to p(y,x)), \forall x, y, z.\, ((p(x,y) \land p(y,z))
      \to p(x,z))\,\} \models \forall x.\, p(x,x).
    \end{equation*}
  \end{enumerate}
\end{exercise}

\begin{exercise}
  % TheoLog 2014
  Formalisieren Sie Bertrand Russells Barbier-Paradoxon
  \begin{quote}
    \emph{Der Barbier rasiert genau diejenigen Personen, die sich nicht selbst rasieren.}
  \end{quote}
  als eine prädikatenlogische Formel und zeigen Sie, dass diese unerfüllbar ist.
\end{exercise}

\end{document}
