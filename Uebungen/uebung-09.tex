% ⓒ 2017 Monika Sturm, Daniel Borchmann
% This work is licensed under the Creative Commons Attribution-ShareAlike 4.0
% International License. To view a copy of this license, visit
% http://creativecommons.org/licenses/by-sa/4.0/.

\documentclass[german]{latteachCD}[2017/03/28]

% Fun with Complexity Theory

\usepackage[sheetnumber=9]{theolog}

\begin{document}

\maketitle

\begin{mdframed}
  Die folgenden Aufgaben werden nicht in den Übungen besprochen und dienen der
  Selbstkontrolle.

  \renewcommand{\theexercise}{\Roman{exercise}}
  \setcounter{exercise}{0}

  \begin{exercise}
    % TheoLog 2014
    Zeigen Sie die folgenden Aussagen:
    \begin{enumerate}
    \item Es gilt $\{\phi\} \models \psi$ genau dann, wenn $\phi \to \psi$
      allgemeingültig ist.
    \item Es gilt $\{\phi_{1}, \dots, \phi_{k}\} \models \psi$ genau dann, wenn
      $\bigwedge_{i=1}^{k} \phi_{i} \to \psi$ allgemeingültig ist.
    \end{enumerate}
  \end{exercise}

  \begin{exercise}
    % TheoLog 2014
    Geben Sie eine erfüllbare Formel in Prädikatenlogik mit Gleichheit an, die
    kein endliches Modell besitzt.
  \end{exercise}

  \begin{exercise}
    % TheoLog 2014
    Seien $\phi$, $\psi$ Formeln und $x$ eine Variable.  Zeigen Sie, dass dann gilt
    \begin{equation*}
      \exists x.(\phi \to \psi) \equiv (\forall x. \phi) \to (\exists x. \psi).
    \end{equation*}
  \end{exercise}
\end{mdframed}

\vspace*{0.5\baselineskip}

\setcounter{exercise}{0}

\begin{exercise}
  % TheoLog 2014
  Geben Sie für die Formel
  \begin{equation*}
    \forall x. \exists y. (P(f(x,y),z) \land (Q(x,g(x),z) \lor P(g(y),y)))
  \end{equation*}
  folgendes an:
  \begin{enumerate}
  \item die Menge aller Unterformeln;
  \item die Menge aller Terme;
  \item die Menge aller Variablen, mit Unterscheidung freier und gebundener
    Variablen;
  \item ein Modell.
  \end{enumerate}
\end{exercise}

\clearpage

\begin{exercise}
  % TheoLog 2014
  Welche der angegebenen Strukturen sind Modelle der folgenden Formel in
  Prädikatenlogik mit Gleichheit?
  \begin{align*}
    \phi \coloneqq{}& \Bigl(\forall x. R(x,x)\Bigr) \land \Bigl(\forall x. \forall y. R(x,y) \land
           R(y,x) \to x = y\Bigr)\\
    &\land \Bigl(\forall x. \forall y. \forall z. R(x,y) \land R(y,z)
      \to R(x,z)\Bigr)
  \end{align*}
  \begin{enumerate}
  \item $\mathcal{A}_{1}$ mit Grundmenge $\mathbb N$ und $R^{\mathcal{A}_{1}} =
    \{\,(m,n) \mid m < n \,\}$;
  \item $\mathcal{A}_{2}$ mit Grundmenge $\mathbb N$ und $R^{\mathcal{A}_{2}} =
    \{\,(n,n+1) \mid n \in \mathbb N \,\}$;
  \item $\mathcal{A}_{3}$ mit Grundmenge $\mathbb N$ und $R^{\mathcal{A}_{3}} =
    \{\,(m,n) \mid m \text{ teilt } n\,\}$;
  \item $\mathcal{A}_{4}$ mit Grundmenge $\Sigma^{*}$ für ein Alphabet $\Sigma$
    und $R^{\mathcal{A}_{4}} = \{\,(x,y) \mid \exists z \in \Sigma^{+}. xz =
    y\,\}$;
  \item $\mathcal{A}_{5}$ mit Grundmenge $\subsets{M}$ für eine Menge $M$ und
    $R^{\mathcal{A}_{5}} = \{\,(X,Y) \mid X \subseteq Y\,\}$;
  \item $\mathcal{A}_{6}$ mit Grundmenge $\mathbb N \times \mathbb N$ und
    \begin{equation*}
      R^{\mathcal{A}_{6}} = \{\,((m_{1},m_{2}),(n_{1},n_{2})) \mid m_{1} < n_{1}
      \lor (m_{1} = n_{1} \land m_{2} \leq n_{2})\,\}.
    \end{equation*}
  \end{enumerate}
\end{exercise}

\begin{exercise}
  \begin{enumerate}
  \item Geben Sie je eine erfüllbare Formel in Prädikatenlogik mit Gleichheit
    an, so dass alle Modelle
    \begin{enumerate}
    \item höchstens drei,
    \item mindestens drei,
    \item genau drei
    \end{enumerate}
    Elemente in der Grundmenge besitzen.
  \item Geben Sie je eine erfüllbare Formel in Prädikatenlogik mit Gleichheit
    an, so dass das einstellige Funktionssymbol $f$ in jedem Modell als eine
    \begin{enumerate}
    \item injektive Funktion,
    \item surjektive Funktion,
    \item bijektive Funktion
    \end{enumerate}
    interpretiert wird.
  \end{enumerate}
\end{exercise}

\begin{exercise}
  Welche der folgenden Aussagen sind wahr?  Begründen Sie Ihre Antwort.
  \begin{enumerate}
  \item Aus $\Gamma \subseteq \Gamma'$ und $\Gamma \models \phi$ folgt $\Gamma'
    \models \phi$.
  \item Jede aussagenlogische Formel ist eine prädikatenlogische Formel.
  \item Eine prädikatenlogische Formel $\phi$ ist genau dann allgemeingültig,
    wenn $\lnot \phi$ unerfüllbar ist.
  \item Es gilt
    \begin{equation*}
      \{\,\forall x. \forall y. R(x,y) \to R(y,x), \forall x. \forall y. \forall
      z. R(x,y) \land R(y,z) \to R(x,z)\,\} \models \forall x. R(x,x).
    \end{equation*}
  \end{enumerate}

\end{exercise}

\end{document}
