% ⓒ 2017 Monika Sturm, Daniel Borchmann
% This work is licensed under the Creative Commons Attribution-ShareAlike 4.0
% International License. To view a copy of this license, visit
% http://creativecommons.org/licenses/by-sa/4.0/.

\documentclass[german]{latteachCD}[2017/03/28]

% PSpace

\usepackage[sheetnumber=7]{theolog}

\begin{document}

\maketitle

\begin{mdframed}
  Die folgenden Aufgaben werden nicht in den Übungen besprochen und dienen der
  Selbstkontrolle.

  \renewcommand{\theexercise}{\BoldGreek{exercise}}
  \setcounter{exercise}{3}

  \begin{exercise}
    % TheoLog 2014
    Zeigen Sie folgende Aussagen:
    \begin{enumerate}
    \item Ist $\Slang{L}_{2} \in \PSpace$ und gilt $\Slang{L}_{1} \leq_{p}
      \Slang{L}_{2}$, so ist auch $\Slang{L}_{1} \in \PSpace$.
    \item Ist $\Slang{L}_{1}$ ein \PSpace-hartes Problem, und gilt
      $\Slang{L}_{1} \leq_{p} \Slang{L}_{2}$, dann ist auch $\Slang{L}_{2}$ ein
      \PSpace-hartes Problem.
    \end{enumerate}
  \end{exercise}

  \begin{exercise}
    Welche der folgenden Aussagen sind wahr?  Begründen Sie Ihre Antwort.
    \begin{enumerate}
    \item Jedes \PSpace-harte Problem ist \NP-hart.
    \item Es gibt kein \NP-hartes Problem, welches in \PSpace{} liegt.
    \item Jedes \NP-vollständige Problem liegt in \PSpace.
    \item Es gilt $\NP = \PSpace$, wenn es ein \PSpace-hartes Problem in \NP{}
      gibt.
    \item Wenn $\PTime \neq \NP$ gilt, dann gibt es kein \NP-hartes Problem in
      \PTime.
    \item Sei $L$ ein \PSpace-vollständiges Problem.  Dann gilt $L \in \PTime
      \iff \PTime = \PSpace$.
    \end{enumerate}
  \end{exercise}
\end{mdframed}

\vspace*{0.5\baselineskip}

\setcounter{exercise}{0}

\begin{exercise}
  % TheoLog 2014
  Zeigen Sie, dass \PSpace{} unter Komplement, Durchschnitt, Vereinigung,
  Konkatenation und Kleene-Stern abgeschlossen ist.
\end{exercise}

\begin{exercise}
  % Complexity Theory 2015/16, originally Sipser 8.26
  Wir betrachten das japanische Spiel \emph{Gomoku}, welches von zwei Spielern
  \textsf{X} und \textsf{O} auf einem $19{\times}\mskip-2mu19$-Brett % spacing
  % is weird; Max dazu: „Space-Complexity und kaputtes Spacing, das passt doch!“
  gespielt wird.  Die Spieler setzen abwechselnd ihre Steine auf das Brett, und
  derjenige Spieler, der zuerst fünf Steine in einer Reihe (horizontal,
  vertikal, oder diagonal) gelegt hat, gewinnt.  Spieler \textsf{X} beginnt.

  \emph{Verallgemeinertes Gomoku} wird statt auf einem Brett fester Größe auf
  einem beliebigen $n\mathord{\times}n$-Brett gespielt.  Eine \emph{Position} in
  diesem Spiel ist eine Belegung der Felder des Spielbretts mit Steinen der
  Spieler \textsf{X} und \textsf{O}, wie sie in einem wirklichen Spiel auftreten
  könnte.  Sei
  \begin{align*}
    \Slang{GM} \coloneqq \{\,\enc(B) \mid \, % again weird spacing …
    & B \text{ ist eine Position im verallgemeinerten Gomoku,} \\
    & \text{in der \textsf{X} eine Gewinnstrategie hat}\,\},
  \end{align*}
  wobei $\enc(B)$ die zeilenweise Kodierung der Position $B$ über einem festen
  Alphabet ist.

  Zeigen Sie $\Slang{GM} \in \PSpace$.
\end{exercise}

\begin{exercise}
  % TheoLog 2014
  Welche der folgenden QBF-Formeln sind erfüllbar?  Begründen Sie Ihre Antwort.
  \begin{enumerate}
  \item $\exists p_{1}. p_{1}$
  \item $\forall p_{1}. p_{1}$
  \item $\exists p_{1}. \bot$
  \item $\forall p_{1}. \exists p_{2}. p_{2} \to p_{1}$
  \item $\forall p_{1}. \exists p_{2}. \forall p_{3}. (p_{1} \lor p_{2}) \land
    p_{3}$
  \item $\forall p_{1}. \forall p_{2}. \exists p_{3}. \forall p_{4}. (p_{1}
    \land p_{2} \to p_{4}) \lor \lnot p_{3}$
  \end{enumerate}

\end{exercise}

\begin{exercise}
  % Complexity Theory 2015/16, originally Sipser 8.29
  Zeigen Sie, dass das Wortproblem für LBA
  \begin{equation*}
    \Slang{P}_{\mathsf{LBA}} \coloneqq \{\,\enc(\Smach{M})\#\#\enc(w) \mid
    \text{\Smach{M} ein LBA, der $w$ akzeptiert} \,\}.
  \end{equation*}
  ein \PSpace-vollständiges Problem ist.
  % TODO: eventuell nochmal erklären, was ein LBA ist (eine TM, die niemals das
  % Bandleerzeichen überschreibt und bei Lesen des Bandleerzeichens nur nach links geht)
\end{exercise}

\end{document}
