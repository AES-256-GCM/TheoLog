% ⓒ 2017 Monika Sturm, Francesco Kriegel, Daniel Borchmann
% This work is licensed under the Creative Commons Attribution-ShareAlike 4.0
% International License. To view a copy of this license, visit
% http://creativecommons.org/licenses/by-sa/4.0/.

\documentclass[german]{latteachCD}[2017/03/28]

% Resolution, wie Blatt 13 von TheoLog 2014

\usepackage[sheetnumber=11]{theolog}

\begin{document}

\maketitle

\begin{mdframed}
  Die folgenden Aufgaben werden nicht in den Übungen besprochen und dienen der
  Selbstkontrolle.

  \renewcommand{\theexercise}{\Roman{exercise}}
  \setcounter{exercise}{5}

  \begin{exercise}
    % TheoLog 2014
    Welche der folgenden Aussagen sind wahr?  Begründen Sie Ihre Antwort.
    \begin{enumerate}
    \item Zwei prädikatenlogische Formeln $F$ und $G$ sind äquivalent, wenn die
      Formel $F \leftrightarrow G$ allgemeingültig ist.
    \item Jede erfüllbare Formel der Prädikatenlogik erster Stufe hat ein
      endliches Modell.
    \item Jede erfüllbare Formel der Prädikatenlogik erster Stufe hat ein
      abzählbares Modell.
    \item Jede Skolemformel hat höchstens eine Herbrand-Interpretation.
    \item Jede Skolemformel hat mindestens ein Herbrand-Modell.
    \end{enumerate}
  \end{exercise}

  \begin{exercise}
    % TheoLog 2014
    Zeigen Sie, dass man das Resolutionsverfahren der Prädikatenlogik erster Stufe
    auch zum Nachweis von semantischen Konsequenzen nutzen kann, indem Sie die
    Äquivalenz der folgenden Aussagen nachweisen:
    \begin{enumerate}
    \item $\Gamma \models F$.
    \item $\Gamma \cup \{\neg F\}$ ist unerfüllbar.
    \item $\bigwedge \Gamma \to F$ ist allgemeingültig.
    \item $\bigwedge \Gamma \land \neg F$ ist unerfüllbar.
    \end{enumerate}
    Hierbei sei $\bigwedge \Gamma = \gamma_{1} \land \dots \land \gamma_{n}$ für
    $\Gamma = \{\gamma_{1},\ldots,\gamma_{n}\}$.
  \end{exercise}
\end{mdframed}

\newpage

\setcounter{exercise}{0}

% Unifikation

\def\deq{\stackrel{\cdot}{=}}

\begin{exercise}
  % TheoLog 2014
  Bestimmen Sie jeweils einen allgemeinsten Unifikator der folgenden
  Gleichungsmengen, oder begründen Sie, warum kein allgemeinster Unifikator
  existiert.  Verwenden Sie hierfür den Algorithmus aus der Vorlesung.  Dabei
  sind $x$, $y$ Variablen und $a$, $b$ Konstanten.
  \begin{enumerate}
  \item $\{\,f(x) \deq g(x,y), y \deq f(a)\,\}$
  \item $\{\,f(g(x,y)) \deq f(g(a,h(b)))\,\}$
  \item $\{\,f(x,y) \deq x, y \deq g(x)\,\}$
  \item $\{\,f(g(x),y) \deq f(g(x),a), g(x) \deq g(h(a))\,\}$
  \end{enumerate}
\end{exercise}

% Resolution

\begin{exercise}
  % TheoLog 2014, teilweise aus Schöning
  Zeigen Sie mittels prädikatenlogischer Resolution folgende Aussagen:
  \begin{enumerate}
  \item Die Aussage \enquote{Der Professor ist glücklich, wenn alle seine
      Studenten Logik mögen} hat als Folgerung \enquote{Der Professor ist
      glücklich, wenn er keine Studenten hat}.
  \item Die Formulierung des Barbier-Paradoxons aus Aufgabe~4 von Blatt~9 ist
    unerfüllbar.
  \item In Aufgabe~V folgt die letzte Aussage aus den ersten drei.  (Zur
    Vereinfachung darf hier angenommen werden, dass alle Individuen Drachen
    sind.)
  \end{enumerate}
\end{exercise}

\end{document}
