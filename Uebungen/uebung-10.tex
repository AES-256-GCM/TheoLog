% ⓒ 2017 Monika Sturm, Francesco Kriegel, Daniel Borchmann
% This work is licensed under the Creative Commons Attribution-ShareAlike 4.0
% International License. To view a copy of this license, visit
% http://creativecommons.org/licenses/by-sa/4.0/.

\documentclass[german]{latteachCD}[2017/03/28]

% Vorbereitung Resolution: Skolem-Normalform, Herbrand-Universum

\usepackage[sheetnumber=10]{theolog}

\begin{document}

\maketitle

\begin{mdframed}
  Die folgenden Aufgaben werden nicht in den Übungen besprochen und dienen der
  Selbstkontrolle.

  \renewcommand{\theexercise}{\Roman{exercise}}
  \setcounter{exercise}{3}

  \begin{exercise}
    % TheoLog 2014
    Welche der folgenden Aussagen sind wahr?  Begründen Sie Ihre Antwort.
    \begin{enumerate}
    \item Jede Formel in Pränexform ist in Skolemform.
    \item Jede Formel in Skolemform ist in Pränexform.
    \item Jede Formel ist äquivalent zu einer bereinigten Formel.
    \item Jede Formel ist äquivalent zu einer bereinigten Formel in Pränexform.
    \item Jede Formel ist äquivalent zu einer bereinigten Formel in Skolemform.
    \end{enumerate}
  \end{exercise}

  \begin{exercise}
    % TheoLog 2014, originally from Schöning
    Formalisieren Sie die folgenden Aussagen in Prädikatenlogik:
    \begin{enumerate}
    \item Jeder Drache ist glücklich, wenn alle seine Kinder fliegen können.
    \item Grüne Drachen können fliegen.
    \item Ein Drache ist grün, wenn er Kind mindestens eines grünen Drachen ist.
    \item Alle grünen Drachen sind glücklich.
    \end{enumerate}
    Zeigen Sie, dass die letzte Aussage aus den ersten drei folgt.
  \end{exercise}
\end{mdframed}

\setcounter{exercise}{0}

\vspace*{\baselineskip}

% Skolem-Normalform

\begin{exercise}
  % TheoLog 2014
  % TODO: Formulierung an Vorlesung anpassen, eventuell Funktionssymbole ersetzen
  Bestimmen Sie zu jeder der folgenden Formeln eine äquivalente bereinigte
  Formel in Pränexform.
  \begin{enumerate}
  \item $\forall x.(p(x,x) \leftrightarrow \lnot \exists y.q(x,y))$
  \item $\forall x.p(f(x,x)) \lor (q(x,z) \to \exists x.p(g(x,y,z)))$
  \item $\forall x.p(x) \land ((\forall y.\exists x.q(x,g(y))) \to \exists
    y.(r(f(y)) \lor \lnot q(y,x)))$
  \end{enumerate}
\end{exercise}

\newpage

\begin{exercise}
  % TheoLog 2014
  Bestimmen Sie zu jeder der folgenden Formeln eine erfüllbarkeitsäquivalente
  bereinigte Formel in Skolemform.
  \begin{enumerate}
  \item $p(x) \lor \exists x.q(x,x) \lor \forall x.p(f(x))$
  \item $\forall x.\exists y.q(f(x),g(y)) \land \forall x.(p(x,y,y) \lor
    q(h(y),x))$
  \item $\forall x.\forall x.(p(x) \leftrightarrow q(x,x)) \lor \exists
    x.\forall y.(q(x,g(y,z)) \land \exists z.q(z,z))$
  \end{enumerate}
\end{exercise}

% Herbrand-Universum

\begin{exercise}
  % TheoLog 2014
  Gegeben sind die folgenden Formeln in Skolemform.
  \begin{align*}
    F &= \forall x, y, z. p(x,f(y),g(z,x)),\\
    G &= \forall x,y. (p(a,f(a,x,y)) \lor q(b)).
  \end{align*}
  \begin{enumerate}
  \item Geben Sie die zugehörigen Herbrand-Universen $\Delta_{F}$ und
    $\Delta_{G}$ an.
  \item Geben Sie je ein Herbrand-Modell an oder begründen Sie, warum kein
    solches existiert.
  \item Geben Sie die Herbrand-Expansion $\operatorname{HE}(F)$ und
    $\operatorname{HE}(G)$ an.
  \end{enumerate}
\end{exercise}

\begin{exercise}
  % TheoLog 2014
  Zeigen Sie, dass Allgemeingültigkeit von Formeln der Prädikatenlogik erster
  Stufe in Skolemform entscheidbar ist.
\end{exercise}

\end{document}
