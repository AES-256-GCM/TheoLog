% ⓒ 2017 Monika Sturm, Daniel Borchmann
% This work is licensed under the Creative Commons Attribution-ShareAlike 4.0
% International License. To view a copy of this license, visit
% http://creativecommons.org/licenses/by-sa/4.0/.

\documentclass[german]{latteachCD}[2017/03/28]

% Fun with Complexity Theory

\usepackage[sheetnumber=8]{theolog}

\begin{document}

\maketitle

\begin{mdframed}
  Die folgenden Aufgaben werden nicht in den Übungen besprochen und dienen der
  Selbstkontrolle.

  \renewcommand{\theexercise}{\BoldGreek{exercise}}
  \setcounter{exercise}{5}

  \begin{exercise}
    % Sipser 8.2
    Wir betrachten folgende Position im Tic-Tac-Toe:
    \begin{equation*}
      \begin{array}{c|c|c}
        \mathsf{X}&& \\\hline
                  &\mathsf{O}& \\\hline
        \mathsf{O}&&\mathsf{X}
      \end{array}
    \end{equation*}
    Angenommen, Spieler \textsf{X} ist am Zug.  Beschreiben Sie eine
    Gewinnstrategie für \textsf{X}.
  \end{exercise}

  \begin{exercise}
    Zeigen Sie, dass für jedes \PSpace-vollständige Problem $\Slang{L}$ auch das
    Komplement $\overline{\Slang{L}}$ ein \PSpace-vollständiges Problem ist.
  \end{exercise}

  \begin{exercise}
    Zeigen Sie: ist $\PTime = \NP$, dann sind alle Sprachen $\Slang L \in \PTime
    \setminus \{\emptyset, \Sigma^{*}\}$ \NP-vollständig.
  \end{exercise}

\end{mdframed}

\vspace*{0.5\baselineskip}

\setcounter{exercise}{0}

\begin{exercise}
  % Fragen aus Vorlesung, Folie 10/27
  Begründen Sie folgende Aussagen:
  \begin{enumerate}
  \item Ist $\PTime = \NP$, dann ist $\NP = \coNP$.
  \item Ist $\PTime \neq \NP$, dann ist $\PTime \neq \coNP$.
  \item Ist $\PTime \neq \NP$, dann ist $\LogSpace \neq \NP$.
  \item Ist $\PTime \neq \NP$, dann ist $\PTime \neq \PSpace$.
  \end{enumerate}
\end{exercise}

\begin{exercise}
  % Sipser 7.30
  Wir betrachten folgendes Scheduling-Problem.  Gegeben sind Prüfungen $P_{1},
  \ldots, P_{k}$ sowie Studierende $S_{1}, \ldots, S_{\ell}$, so dass jede
  Prüfung von einer bestimmten Menge von Studierenden abgelegt wird.  Die
  Aufgabe ist, die Prüfungen so zu legen, dass kein Studierende zur gleichen
  Zeit zwei Prüfungen ablegen muss.

  Formalisieren Sie die Frage, ob solch ein Prüfungsplan mit höchstens $h$
  Zeitslots möglich ist, als eine formale Sprache und zeigen Sie, dass diese
  \NP-vollständig ist.
\end{exercise}

\begin{exercise}
  % Sipser 7.27
\end{exercise}

\end{document}
