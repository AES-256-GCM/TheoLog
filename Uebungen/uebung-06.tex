% ⓒ 2017 Monika Sturm, Daniel Borchmann
% This work is licensed under the Creative Commons Attribution-ShareAlike 4.0
% International License. To view a copy of this license, visit
% http://creativecommons.org/licenses/by-sa/4.0/.

\documentclass[german]{latteachCD}[2017/03/28]

% NP-Vollständigkeit

\usepackage[sheetnumber=6]{theolog}

\begin{document}

\maketitle

\begin{mdframed}
  Die folgenden Aufgaben werden nicht in den Übungen besprochen und dienen der
  Selbstkontrolle.

  \renewcommand{\theexercise}{\BoldGreek{exercise}}
  \setcounter{exercise}{1}

  \begin{exercise}
    % Sipser 7.43
    Zeigen Sie, dass \NP{} unter Kleene-Stern abgeschlossen ist.
  \end{exercise}

  \begin{exercise}
    % TheoLog 2014
    Wir betrachten das folgende Problem $K$: Gegeben sind zwei gerichtete
    Graphen $G_{1} = (V_{1}, E_{1})$ und $G_{2} = (V_{2}, E_{2})$ sowie eine
    Zahl $k \in \mathbb N$.  Gefragt ist, ob es Teilmengen $V_{1}' \subseteq
    V_{1}$ und $V_{2}' \subseteq V_{2}$ gibt, so dass $\lvert V_{1}'\rvert =
    \lvert V_{2}'\rvert = k$ ist und es eine Bijektion $f \colon V_{1}' \to
    V_{2}'$ gibt, so dass gilt
    \begin{equation*}
      (u,v) \in E_{1} \iff (f(u), f(v)) \in E_{2}.
    \end{equation*}
    \begin{enumerate}
    \item Zeigen Sie $K \in \NP$.
    \item Zeigen Sie, dass $K$ ein \NP-hartes Problem ist.  Zeigen Sie dafür,
      dass das Problem CLIQUE auf $K$ in polynomieller Zeit reduzierbar ist.
    \end{enumerate}
  \end{exercise}

\end{mdframed}

\vspace*{0.5\baselineskip}

\setcounter{exercise}{0}

\begin{exercise}
  % TheoLog 2014
  Wir betrachten das folgende Problem $K$: Gegeben eine aussagenlogische
  Formel $\phi$ mit $n$-Variablen, gibt es eine erfüllbare Belegung von $\phi$,
  bei der mindestens die Hälfte aller in $\phi$ vorkommenden Variablen mit
  \enquote{true} belegt sind?
  \begin{enumerate}
  \item Formalisieren Sie dieses Problem als Sprache und zeigen Sie, dass $K \in
    \NP$ gilt.
  \item Zeigen Sie, dass $K$ ein \NP-hartes Problem ist.
  \end{enumerate}
\end{exercise}

\begin{exercise}
  % Sipser 7.28
  Im folgenden \emph{Solitaire}-Spiel haben wir ein Spielbrett der Größe $m
  \times m$ gegeben.  Auf jeder der $m^{2}$ Positionen liegt entweder ein blauer
  Stein, ein roter Stein, oder gar nichts.  Das Spiel wird nun so gespielt, dass
  Steine vom Brett genommen werden bis in jeder Spalte nur noch Steine einer
  Farbe liegen, und in jeder Zeile mindestens ein Stein liegen bleibt.  In
  diesem Fall ist das Spiel gewonnen.  Es ist möglich, dass man ausgehend von
  einer gegebenen Anfangsbelegung das Spiel nicht gewinnen kann.

  \begin{enumerate}
  \item Formalisieren Sie das Problem, für eine gegebene Anfangsbelegung im
    Solitaire-Spiel zu entscheiden, ob es möglich ist, das Spiel zu gewinnen,
    als ein Entscheidungsproblem SOLITAIRE.
  \item Zeigen Sie, dass $\text{SOLITAIRE} \in \NP$ gilt.
  \item Zeigen Sie, dass SOLITAIRE ein \NP-hartes Problem ist, indem Sie zeigen,
    dass 3SAT in polynomieller Zeit auf SOLITAIRE reduzierbar ist.
  \end{enumerate}
\end{exercise}

\begin{exercise}
  % Folklore
  Sei $\Sigma$ ein Alphabet und seien $A, B \subseteq \Sigma^{*}$.  Wir sagen,
  dass $A$ auf $B$ in \emph{logarithmischen Platz reduzierbar ist}, und
  schreiben $A \leq_{\ell} B$, falls es eine Many-One-Reduktion von $A$ nach $B$
  gibt, die in logarithmischen Platz berechenbar ist.  Zeigen Sie: gilt $A
  \leq_{\ell} B$ und $B \leq_{\ell} C$, dann gilt auch $A \leq_{\ell} C$.
\end{exercise}

\begin{exercise}
  % TheoLog 2014
  Wir betrachten das Problem SET-SPLITTING, welches für eine gegebene endliche
  Menge $S$ und eine Menge $C = \{C_{1}, \dots, C_{k}\}$ von Teilmengen von $S$
  fragt, ob die Elemente von $S$ derart mit den Farben blau oder rot gefärbt
  werden können, so dass niemals alle Elemente einer Menge $C_{i}$ die gleiche
  Farbe bekommen.  Zeigen Sie, dass SET-SPLITTING ein \NP-vollständiges Problem
  ist.
\end{exercise}

\end{document}

%  LocalWords:  Anfangsbelegung
