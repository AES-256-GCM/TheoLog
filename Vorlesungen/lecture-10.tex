\documentclass[aspectratio=1610,onlymath]{beamer}
% \documentclass[aspectratio=1610,onlymath,handout]{beamer}

% Macros used by all lectures, but not necessarily by excercises

%%% General setup and dependencies:

% \usetheme[ddcfooter,nosectionnum]{tud}
\usetheme[nosectionnum,pagenum,noheader]{tud}
% \usetheme[nosectionnum,pagenum]{tud}

% Increase body font size to a sane level:
\let\origframetitle\frametitle
% \renewcommand{\frametitle}[1]{\origframetitle{#1}\normalsize}
\renewcommand{\frametitle}[1]{\origframetitle{#1}\fontsize{10pt}{13.2}\selectfont}
\setbeamerfont{itemize/enumerate subbody}{size=\small} % tud defaults to scriptsize!
\setbeamerfont{itemize/enumerate subsubbody}{size=\small}
% \setbeamerfont{normal text}{size=\small}
% \setbeamerfont{itemize body}{size=\small}

\renewcommand{\emph}[1]{\textbf{#1}}

\def\arraystretch{1.3}% Make tables even less cramped vertically

\usepackage[ngerman]{babel}
\usepackage[utf8]{inputenc}
\usepackage[T1]{fontenc}

%\usepackage{graphicx}
\usepackage[export]{adjustbox} % loads graphicx
\usepackage{import}
\usepackage{stmaryrd}
\usepackage[normalem]{ulem} % sout command
% \usepackage{times}
\usepackage{txfonts}
\usepackage{array}

% \usepackage[perpage]{footmisc} % reset footnote counter on each page -- fails with beamer (footnotes gone)
\usepackage{perpage}  % reset footnote counter on each page
\MakePerPage{footnote}

\usepackage{tikz}
\usetikzlibrary{arrows,positioning,decorations.pathreplacing}
% Inspired by http://www.texample.net/tikz/examples/hand-drawn-lines/
\usetikzlibrary{decorations.pathmorphing}
\pgfdeclaredecoration{penciline}{initial}{
    \state{initial}[width=+\pgfdecoratedinputsegmentremainingdistance,
    auto corner on length=1mm,]{
        \pgfpathcurveto%
        {% From
            \pgfqpoint{\pgfdecoratedinputsegmentremainingdistance}
                      {\pgfdecorationsegmentamplitude}
        }
        {%  Control 1
        \pgfmathrand
        \pgfpointadd{\pgfqpoint{\pgfdecoratedinputsegmentremainingdistance}{0pt}}
                    {\pgfqpoint{-\pgfdecorationsegmentaspect
                     \pgfdecoratedinputsegmentremainingdistance}%
                               {\pgfmathresult\pgfdecorationsegmentamplitude}
                    }
        }
        {%TO
        \pgfpointadd{\pgfpointdecoratedinputsegmentlast}{\pgfpoint{1pt}{1pt}}
        }
    }
    \state{final}{}
}
\tikzset{handdrawn/.style={decorate,decoration=penciline}}
\tikzset{every shadow/.style={fill=none,shadow xshift=0pt,shadow yshift=0pt}}
% \tikzset{module/.append style={top color=\col,bottom color=\col}}

% Use to make Tikz attributes with Beamer overlays
% http://tex.stackexchange.com/a/6155
\tikzset{onslide/.code args={<#1>#2}{%
  \only<#1| handout:0>{\pgfkeysalso{#2}}
}}
\tikzset{onslideprint/.code args={<#1>#2}{%
  \only<#1>{\pgfkeysalso{#2}}
}}

%%% Title -- always set this first

\newcommand{\defineTitle}[3]{
	\newcommand{\lectureindex}{#1}
	\title{Theoretische Informatik und Logik}
	\subtitle{\href{\lectureurl}{#1. Vorlesung: #2}}
	\author{\href{https://iccl.inf.tu-dresden.de/web/Markus_Kr\%C3\%B6tzsch}{Markus Kr\"{o}tzsch}\\[1ex]Lehrstuhl Wissensbasierte Systeme}
	\date{#3}
	\datecity{TU Dresden}
% 	\institute{CC-By 3.0, sofern keine anderslautenden Bildrechte angegeben sind}
}

%%% Table of contents:

\RequirePackage{ifthen}

\newcommand{\highlight}[2]{%
	\ifthenelse{\equal{#1}{\lectureindex}}{\alert{#2}}{#2}%
}

\def\myspace{-0.7ex}
\newcommand{\printtoc}{
\begin{tabular}{r@{$\quad$}l}
\highlight{1}{1.} & \highlight{1}{Willkommen/Einleitung formale Sprachen}\\[\myspace]
\highlight{2}{2.} & \highlight{2}{Grammatiken und die Chomsky-Hierarchie}\\[\myspace]
\highlight{3}{3.} & \highlight{3}{Endliche Automaten}\\[\myspace]
\highlight{4}{4.} & \highlight{4}{Complexity of FO query answering}\\[\myspace]
\highlight{5}{5.} & \highlight{5}{Conjunctive queries}\\[\myspace]
\highlight{6}{6.} & \highlight{6}{Tree-like conjunctive queries}\\[\myspace]
\highlight{7}{7.} & \highlight{7}{Query optimisation}\\[\myspace]
\highlight{8}{8.} & \highlight{8}{Conjunctive Query Optimisation / First-Order~Expressiveness}\\[\myspace]
\highlight{9}{9.} & \highlight{9}{First-Order~Expressiveness / Introduction to Datalog}\\[\myspace]
\highlight{10}{10.} & \highlight{10}{Expressive Power and Complexity of Datalog}\\[\myspace]
\highlight{11}{11.} & \highlight{11}{Optimisation and Evaluation of Datalog}\\[\myspace]
\highlight{12}{12.} & \highlight{12}{Evaluation of Datalog (2)}\\[\myspace]
\highlight{13}{13.} & \highlight{13}{Graph Databases and Path Queries}\\[\myspace]
\highlight{14}{14.} & \highlight{14}{Outlook: database theory in practice}
\end{tabular}
}

\newcommand{\overviewslide}{%
\begin{frame}\frametitle{Overview}
\printtoc
\medskip

Siehe \href{\lectureurl}{course homepage [$\Rightarrow$ link]} for more information and materials
\end{frame}
}

%%% Colours:
\usepackage{xcolor,colortbl}
\definecolor{redhighlights}{HTML}{FFAA66}
\definecolor{lightblue}{HTML}{55AAFF}
\definecolor{lightred}{HTML}{FF5522}
\definecolor{lightpurple}{HTML}{DD77BB}
\definecolor{lightgreen}{HTML}{55FF55}
\definecolor{darkred}{HTML}{CC4411}
\definecolor{darkblue}{HTML}{176FC0}%{1133AA}
\definecolor{nightblue}{HTML}{2010A0}%{1133AA}
\definecolor{alert}{HTML}{176FC0}
\definecolor{darkgreen}{HTML}{36AB14}
\definecolor{strongyellow}{HTML}{FFE219}
\definecolor{devilscss}{HTML}{666666}

\newcommand{\redalert}[1]{\textcolor{darkred}{#1}}

%%% Slide layout commands:

\newcommand{\sectionSlide}[1]{
\frame{\begin{center}
\LARGE
#1
\end{center}}
}
\newcommand{\sectionSlideNoHandout}[1]{
\frame<handout:0>{\begin{center}
\LARGE
#1
\end{center}}
}

\newcommand{\mydualbox}[3]{%
 \begin{minipage}[t]{#1}
 \begin{beamerboxesrounded}[upper=block title,lower=block body,shadow=true]%
    {\centering\usebeamerfont*{block title}#2}%
    \raggedright%
    \usebeamerfont{block body}
%     \small
    #3%
  \end{beamerboxesrounded}
  \end{minipage}
}
%
\newcommand{\myheaderbox}[2]{%
 \begin{minipage}[t]{#1}
 \begin{beamerboxesrounded}[upper=block title,lower=block title,shadow=true]%
    {\centering\usebeamerfont*{block title}\rule{0pt}{2.6ex} #2}%
  \end{beamerboxesrounded}
  \end{minipage}
}

\newcommand{\mycontentbox}[2]{%
 \begin{minipage}[t]{#1}%
 \begin{beamerboxesrounded}[upper=block body,lower=block body,shadow=true]%
    {\centering\usebeamerfont*{block body}\rule{0pt}{2.6ex}#2}%
  \end{beamerboxesrounded}
  \end{minipage}
}

\newcommand{\mylcontentbox}[2]{%
 \begin{minipage}[t]{#1}%
 \begin{beamerboxesrounded}[upper=block body,lower=block body,shadow=true]%
    {\flushleft\usebeamerfont*{block body}\rule{0pt}{2.6ex}#2}%
  \end{beamerboxesrounded}
  \end{minipage}
}

% label=180:{\rotatebox{90}{{\footnotesize\textcolor{darkgreen}{Beispiel}}}}
% \hspace{-8mm}\ghost{\raisebox{-7mm}{\rotatebox{90}{{\footnotesize\textcolor{darkgreen}{Beispiel}}}}}\hspace{8mm}
\newcommand{\examplebox}[1]{%
	\begin{tikzpicture}[decoration=penciline, decorate]
		\pgfmathsetseed{1235}
		\node (n1) [decorate,draw=darkgreen, fill=darkgreen!10,thick,align=left,text width=\linewidth, inner ysep=2mm, inner xsep=2mm] at (0,0) {#1};
% 		\node (n2) [align=left,text width=\linewidth,inner sep=0mm] at (n1.92) {{\footnotesize\raisebox{3mm}{\textcolor{darkgreen}{Beispiel}}}};
% 		\node (n2) [decorate,draw=darkgreen, fill=darkgreen!10,thick, align=left,text width=\linewidth,inner sep=2mm] at (n1.90) {{\footnotesize\raisebox{0mm}{\textcolor{darkgreen}{Beispiel}}}};
	\end{tikzpicture}%
}%

\newcommand{\codebox}[1]{%
	\begin{tikzpicture}[decoration=penciline, decorate]
		\pgfmathsetseed{1236}
		\node (n1) [decorate,draw=strongyellow, fill=strongyellow!10,thick,align=left,text width=\linewidth, inner ysep=2mm, inner xsep=2mm] at (0,0) {#1};
	\end{tikzpicture}%
}%

\newcommand{\defbox}[1]{%
	\begin{tikzpicture}[decoration=penciline, decorate]
		\pgfmathsetseed{1237}
		\node (n1) [decorate,draw=darkred, fill=darkred!10,thick,align=left,text width=\linewidth, inner ysep=2mm, inner xsep=2mm] at (0,0) {#1};
	\end{tikzpicture}%
}%

\newcommand{\theobox}[1]{%
	\begin{tikzpicture}[decoration=penciline, decorate]
		\pgfmathsetseed{1240}
		\node (n1) [decorate,draw=darkblue, fill=darkblue!10,thick,align=left,text width=\linewidth, inner ysep=2mm, inner xsep=2mm] at (0,0) {#1};
	\end{tikzpicture}%
}%

\newcommand{\anybox}[2]{%
	\begin{tikzpicture}[decoration=penciline, decorate]
		\pgfmathsetseed{1240}
		\node (n1) [decorate,draw=#1, fill=#1!10,thick,align=left,text width=\linewidth, inner ysep=2mm, inner xsep=2mm] at (0,0) {#2};
	\end{tikzpicture}%
}%


\newsavebox{\mybox}%
\newcommand{\doodlebox}[2]{%
\sbox{\mybox}{#2}%
	\begin{tikzpicture}[decoration=penciline, decorate]
		\pgfmathsetseed{1238}
		\node (n1) [decorate,draw=#1, fill=#1!10,thick,align=left,inner sep=1mm] at (0,0) {\usebox{\mybox}};
	\end{tikzpicture}%
}%


\defineTitle{10}{NP, Teil 2}{14. Mai 2018}

\begin{document}

\maketitle

\begin{frame}\frametitle{\Scomplclass{NP}-vollständige Probleme}

\Scomplclass{NP}-vollständige Probleme = Probleme, die mindestens so schwer
sind, wie alle anderen Probleme in \Scomplclass{NP}\bigskip


Bisher haben wir erfahren (und teilweise auch gezeigt):

\begin{center}
{\large Wortproblem Polyzeit-NTM}\\
$\leq_p$\\
{\large\Slang{SAT}}\\
$\leq_p$\\
{\large\Slang{CLIQUE}}\\
$\leq_p$\\
{\large\Slang{Unabhängige Menge}}
\end{center}

\end{frame}

\sectionSlide{Probleme mit Gewichten und Zahlen}

\newcommand{\val}[1]{v(#1)}

\begin{frame}\frametitle{Teilmengen-Summe}

\defbox{\Slang{Teilmengen-Summe} (subset sum)\\[1ex]
\emph{Gegeben:} Eine Menge von Gegenständen $S=\{a_1,\ldots,a_n\}$, wobei jedem Gegenstand $a_i$ ein Wert $\val{a_i}$ zugeordnet ist; eine gewünschte Zahl $z$\\[1ex]
\emph{Frage:} Gibt es eine Teilmenge $T\subseteq S$ mit $\sum_{a\in T} \val{a} = z$?\\
}\smallskip

\emph{Anmerkung:} Mehrere Gegenstände können gleiche Werte haben.\bigskip\pause

\theobox{Satz: \Slang{Teilmengen-Summe} ist \Scomplclass{NP}-vollständig.}\pause

\emph{Beweis:}
\begin{enumerate}[(1)]
\item $\Slang{Teilmengen-Summe}\in\Scomplclass{NP}$ haben wir bereits festgestellt
\item \Scomplclass{NP}-Härte zeigen wir durch Reduktion $\Slang{SAT}\leq_p\Slang{Teilmengen-Summe}$
\end{enumerate}

\end{frame}

\newcommand{\red}{\color{darkred}}

\begin{frame}[label=subsetsum]\frametitle{Beispiel}
  \begin{center}
    {\color{black} \large
      $(p_1\vee p_2\vee p_3)\wedge (\neg p_1 \vee \neg p_4) \wedge (p_4 \vee p_5 \vee \neg p_2 \vee \neg p_3)$
    }

    \vspace{0.05\textheight}
    {
      \color{black}
      \renewcommand{\arraystretch}{0.5}
      \begin{tabular}{ll *{5}{@{\,}l}*{3}{@{\,}c}}
        && $\red{}p_1$   & $\red{}p_2$    &  $\red{}p_3$   & $\red{}p_4$  & $\red{}p_5$     &  \color{blue} $C_1$ & \color{blue} $C_2$ & \color{blue}$C_3$ \\
        \\
        $\val{t_{1}}$ & =~~  & 1 & 0 & 0 & 0 & 0 & \color{blue} 1 & \color{blue} 0 & \color{blue} 0\\
        $\val{f_{1}}$ & =   & 1 & 0 & 0 & 0 & 0 &  \color{blue} 0 & \color{blue} 1& \color{blue} 0\\
        $\val{t_{2}}$ & =  & ~ & 1 & 0 & 0 & 0 &  \color{blue} 1 & \color{blue} 0 & \color{blue} 0\\
        $\val{f_{2}}$ & = & ~ & 1 & 0 & 0 & 0 &  \color{blue} 0 & \color{blue} 0 & \color{blue} 1\\
        $\val{t_{3}}$ & =  & ~ & ~ & 1 & 0 & 0 &  \color{blue} 1 & \color{blue} 0 & \color{blue} 0\\
        $\val{f_{3}}$ & = & ~ & ~ & 1 & 0 & 0 &  \color{blue} 0 & \color{blue} 0 & \color{blue}1\\
        $\val{t_{4}}$ & =  & ~ & ~ & ~ & 1 & 0 &  \color{blue} 0 & \color{blue} 0 & \color{blue}1\\
        $\val{f_{4}}$ & = & ~ & ~ & ~ & 1 & 0 &  \color{blue} 0 & \color{blue}1 & \color{blue}0\\
        $\val{t_{5}}$ & =  & ~ & ~ & ~ & ~ & 1 &  \color{blue}0 &\color{blue} 0 & \color{blue}1\\
        $\val{f_{5}}$ & = & ~ & ~ & ~ & ~ & 1 &  \color{blue}0 & \color{blue}0 & \color{blue} 0\\
        \\
        $\val{m_{1,1}}$ & = & ~ & ~ & ~ & ~ & ~ &  1 & 0 & 0\\
        $\val{m_{1,2}}$ & = & ~ & ~ & ~ & ~ & ~ &  1 & 0 & 0\\
        $\val{m_{2,1}}$ & = & ~ & ~ & ~ & ~ & ~ &  ~ & 1 & 0\\
        $\val{m_{3,1}}$ & = & ~ & ~ & ~ & ~ & ~ &  ~ & ~ & 1\\
        $\val{m_{3,2}}$ & = & ~ & ~ & ~ & ~ & ~ &  ~ & ~ & 1\\
        $\val{m_{3,3}}$ & = & ~ & ~ & ~ & ~ & ~ & ~ & ~ & 1\\
        \hline\\
        $z$       & = & 1 & 1 & 1 & 1 & 1 & 3 & 2 & 4
      \end{tabular}
    }
  \end{center}
\end{frame}

\begin{frame}
  \frametitle{\Slang{SAT} $\leq_p$ \Slang{Teilmengen-Summe}}
    \emph{Gegeben: } Formel $F = (C_1 \wedge \dots \wedge C_k)$ in
  CNF.\\
  (o.B.d.A. mit maximal 9 Literalen pro Klausel)\pause\\[1ex]
  Seien $p_1, \dots, p_n$ die Variablen in $F$.\\
  Für jedes $p_i$ definieren wir Gegenstände $t_i$ und $f_i$:\\[10pt]

  $\begin{array}{ll}
    \val{t_i} := a_1 \cdots a_n c_1 \cdots c_k & {\text{wobei}}
    \begin{array}{l}
      a_j :=
      \begin{cases}
        1 & i=j\\
        0 & i\not=j
      \end{cases}\\[15pt]
      c_j :=
      \begin{cases}
        1 & p_i \text{ kommt in } C_j\text{ vor}\\
        0 & \text{sonst}
      \end{cases}
    \end{array}
  \end{array}$

  \bigskip\pause

  $\begin{array}{ll}
    \val{f_i} := a_1 \cdots a_n c_1 \cdots c_k & \text{wobei} 
    \begin{array}{l}
      a_j :=
      \begin{cases}
        1 & i=j\\
        0 & i\not=j
      \end{cases}\\[15pt]
      c_j :=
      \begin{cases}
        1 & {\neg p_i} \text{ kommt in } C_j\text{ vor}\\
        0 & \text{sonst}
      \end{cases}
    \end{array}
  \end{array}$
\end{frame}

\againframe{subsetsum}

\begin{frame}
  \frametitle{\Slang{SAT} $\leq_p$ \Slang{Teilmengen-Summe}}

  Außerdem definieren wir für jede Klausel $C_i$ genau $r := |C_i|-1$ Gegenstände
  $m_{i,1}, \ldots, m_{i,r}$\\[5pt]
  mit $\quad\val{m_{i, j}} := c_i \cdots c_k\quad$ wobei
  $\quad
  c_\ell :=
  \begin{cases}
    1 & \ell=i\\
    0 & \ell\not= i
  \end{cases}
  $\bigskip\pause

  \emph{Definition von $S$: } Damit ergibt sich die Menge
  \[
     S := \{ t_i, f_i \mid 1\leq i \leq
       n\} \cup \{ m_{i,j} \mid 1\leq i\leq k,~~ 1\leq j \leq |C_i|-1\}
  \]

  \pause\emph{Gesuchte Zahl: } Wir bestimmten die gesuchte Zahl $z$ wie folgt:\\
  \[
     \text{$z := a_1 \cdots a_n c_1 \cdots c_k$ wobei $a_i := 1$ und $c_i := |C_i|$ }
  \]

  \bigskip

  \pause\emph{Behauptung: } Es gibt $T\subseteq S$ mit $\sum_{a_i\in T}
    \val{a_i} = z$ gdw. $F$ erfüllbar ist.
\end{frame}

\againframe{subsetsum}

\begin{frame}\frametitle{NP-Vollständigkeit von \textsc{Teilmengen-Summe}}

\anybox{lightblue}{
  \emph{Behauptung: }\\
  Wenn $F$ erfüllbar ist, dann gibt es $T\subseteq S$ mit $\sum_{a_i\in T}
    \val{a_i} = z$.}\bigskip

%
\pause
\begin{itemize}
\item Sei $w$ eine erfüllende Belegung für $F$\pause\\[2ex]
%
\item Wir definieren: 
    \[T_1 :=
		\{ t_i \mid w(p_i) = \mytrue,~ 1\leq i \leq m\}\
			\cup \{ f_i \mid w(p_i) = \myfalse,~ 1\leq i \leq m\}
    \]\pause%
\item Sei $r_i$ jeweils die Zahl der von $w$ erfüllten Literale in \ghost{$C_i$.}\\
Wir definieren:
\[T_2 := \{ m_{i,j}\mid 1\leq i \leq k,~ 1\leq j \leq |C_i|-r_i \}\]\pause%
% 
\item Die gesuchte Menge von Gegenständen ist $T := T_1 \cup T_2$.\\[5pt]
Damit folgt: $\sum_{s\in T} \val{s} = z$
\end{itemize}
\end{frame}


\begin{frame}
  \frametitle{NP-Vollständigkeit von \textsc{Teilmengen-Summe}}

  \anybox{lightblue}{
    \emph{Behauptung:}\\ Gibt es $T\subseteq S$ mit $\sum_{a_i\in T}
    \val{a_i} = z$ dann ist $F$ erfüllbar.}\bigskip\pause

  Sei $T \subseteq S$ die gesuchte Menge mit $\sum_{s\in T} \val{s} = z$\pause\\[10pt]
%
  Wir definieren $w(p_i) =
    \begin{cases}
      \mytrue & \text{ falls } t_i \in T\\
      \myfalse & \text{ falls } f_i \in T
    \end{cases}$\\[10pt]
%
  $w$ ist wohldefiniert, da für alle $i$ gilt:\\ $t_i \in T$ oder
  $f_i \in T$, aber nicht beides\pause\\[10pt]
%
  Außerdem muss für jede Klausel ein Literal auf $\mytrue$ abgebildet werden,
  da die entsprechenden Hilfszahlen $m_{i,j} \in S$ zusammen nicht die
  geforderte Zahl an Literalen pro Klausel erreichen.
\qed

\end{frame}

\newcommand{\weight}[1]{g(#1)}

\begin{frame}\frametitle{Das Rucksack-Problem}

\defbox{\Slang{Rucksack} (Knapsack)\\[1ex]
\emph{Gegeben:} Eine Menge von Gegenständen $G=\{a_1,\ldots,a_n\}$, wobei jedem Gegenstand $a_i$ ein Wert $\val{a_i}$ und ein Gewicht $\weight{a_i}$ zugeordnet ist; ein Mindestwert $w$ und ein Gewichtslimit $\ell$\\[1ex]
\emph{Frage:} Gibt es eine Teilmenge $T\subseteq G$, so dass
\begin{itemize}
\item $\sum_{a\in T} \weight{a} \leq \ell$ und
\item $\sum_{a\in T} \val{a} \geq w$~?
\end{itemize}
}\smallskip\pause

\theobox{Satz: \Slang{Rucksack} ist \Scomplclass{NP}-vollständig.}\pause

\emph{Beweis:}
\begin{enumerate}[(1)]
\item $\Slang{Rucksack} \in \Scomplclass{NP}$: Das Zertifikat ist $T$
\item \Slang{Rucksack} ist \Scomplclass{NP}-hart:\\
durch Reduktion $\Slang{Teilmengen-Summe}\leq_p\Slang{Rucksack}$
\end{enumerate}

\end{frame}


\begin{frame}\frametitle{$\Slang{Teilmengen-Summe}\leq_p\Slang{Rucksack}$}

\emph{Gegeben:} Eine Instanz von \Slang{Teilmengen-Summe} (Menge von Gegenständen $S=\{a_1,\ldots,a_n\}$ mit Wert $\val{a_i}$; gewünschte Zahl $z$)\pause

  \bigskip

  \emph{Daraus konstruieren wir:}

  \begin{itemize}
  \item $G := \{ 1, \dots, n\}$: Menge der Gegenstände
  \item Für alle $i\in G$ setzen wir: $\val{i} = \weight{i} = \val{a_i}$
  \item Zielwert $w := z$ und Gewichtslimit $\ell := z$
  \end{itemize}
  Offensichtlich ist diese Übersetzung \alert{polynomiell.}\medskip\pause

  Damit gilt für jede Teilmenge $T\subseteq G$
  \[
  \sum_{i\in T}\val{a_i}   = z\qquad {gdw.}\qquad
  \begin{array}{l}
    \sum_{i\in T} \val{i}    \geq w    \quad{\red = z}\\[5pt]
    \sum_{i\in T} \weight{i} \leq \ell \quad{\red = z}
  \end{array}
  \]
  $\leadsto$ Die Reduktion ist \alert{korrekt.}\qed

\end{frame}

\sectionSlide{Pseudopolynomielle Probleme}

\begin{frame}<1>[label=knapsack]
  \frametitle{Eine polynomielle Lösung für \Slang{Rucksack}}
  
  Mittels \alert{dynamischer Programmierung} kann man \Slang{Rucksack}
  in der Zeit $O(n\ell)$ lösen\\[10pt]
%
  \emph{Initialisierung:} \\
   Erzeuge eine $(\ell+1) \times (n+1)$-Matrix $M$\\[5pt]
%
   Setze $M(g, 0) = M(0,i) =  0$ \qquad für alle
   $1\leq g \leq \ell$ \quad  $1\leq i\leq n$\\[10pt]\pause
%
  \emph{Berechnung: }
   Für $i = 0,1, \dots, n-1$ berechne $M(g, i+1)$ als\\[1ex]
% 
	\narrowcentering{$M(g, i+1) := \max\big( M(g, i),~~ M(g-\weight{i+1}, i) + \val{i+1}\big)$}\\[1ex]
% 
    (Falls $g-\weight{i+1} < 0$, dann verwenden wir immer $M(g, i)$.)\\[5pt]
%
    $M(g, i)$ = Größter Gesamtwert unter den ersten $i$ Gegenständen\\
    \mbox{}\phantom{$M(g, i)$ =} bei Einhaltung des Gewichtslimits $g$\\[5pt]
%
	\emph{Akzeptanz: } Akzeptiere falls $M$ einen Eintrag $\geq w$ hat.

\end{frame}

\newcommand{\blue}{\color{darkblue}}
\newcommand{\black}{\color{black}}

\begin{frame}<1-2>[label=knapsack-ex]
  \frametitle{Beispiel}

  \emph{Eingabe: }
  $G := \{ 1, 2, 3, 4\}$ mit \\[5pt]
  $\begin{array}{lcccc}
    \text{Werten:}\blue &\blue  \val{1} := 1 &\blue  \val{2} := 3 &\blue  \val{3}
    := 4 &\blue  \val{4} := 2\\[-0.5ex]
    \text{Gewichten: }\blue &\blue \weight{1} := 1 &\blue \weight{2} := 1 &\blue \weight{3} := 3 &\blue \weight{4}:= 2
  \end{array}$    \\[5pt]

  \emph{Gewichtslimit: } $\ell := 5$\hspace{2cm}
  \emph{Mindestwert: } $w := 7$

  {\footnotesize
  \[
  \begin{array}{|c|c|c|c|c|c|}\hline
    \text{Gewichts-} & \multicolumn{5}{c|}{\text{max. Wert für $\leq i$ Gegenstände} }\\\cline{2-6}
    \text{limit } {\blue g} & {\blue{}i=0}& {\blue{}i=1}& {\blue{}i=2}&
    {\blue{}i=3}& {\blue{}i=4} \\\hline
    {\blue{}0} & \uncover<2->0& \uncover<2->0& \uncover<2->0& \uncover<2->0& \uncover<2->0\\\hline
    {\blue{}1} & \uncover<2->0& \uncover<4->1& \uncover<5->3& \uncover<10->3& \uncover<10->3\\\hline
    {\blue{}2} & \uncover<2->0& \uncover<4->1& \uncover<6->4& \uncover<10->4& \uncover<10->4\\\hline
    {\blue{}3} & \uncover<2->0& \uncover<4->1& \uncover<7->4& \uncover<10->4& \uncover<10->5\\\hline
    {\blue{}4} & \uncover<2->0& \uncover<4->1& \uncover<8->4& \uncover<10->7& \uncover<10->7\\\hline
    {\blue{}5} & \uncover<2->0& \uncover<4->1& \uncover<9->4& \uncover<10->8& \uncover<10->8\\\hline
  \end{array}
  \]}
\only<1-2>{Initialisiere $M(g, 0) = M(0,i) =  0$ für alle
	$1\leq g \leq \ell$ und $1\leq i\leq n$}
\only<3->{\narrowcentering{$M(g, i+1) := \max\big( M(g, i),~~ M(g-\weight{i+1}, i) + \val{i+1}\big)$}}
\end{frame}


\againframe{knapsack}

\againframe<3->{knapsack-ex}

\begin{frame}\frametitle{$\Scomplclass{P}=\Scomplclass{NP}$?}

Zusammenfassung:
\begin{itemize}
\item \Slang{Rucksack} ist \Scomplclass{NP}-vollständig
\item \Slang{Rucksack} ist mittels Dynamischer Programmierung in Zeit $O(n\ell)$ lösbar
\end{itemize}

\defbox{\Slang{Rucksack} (Knapsack)\\[1ex]
\emph{Gegeben:} Eine Menge von Gegenständen $G=\{a_1,\ldots,a_n\}$, wobei jedem Gegenstand $a_i$ ein Wert $\val{a_i}$ und ein Gewicht $\weight{a_i}$ zugeordnet ist; ein Mindestwert $w$ und ein Gewichtslimit $\ell$\\[1ex]
\emph{Frage:} Gibt es eine Teilmenge $T\subseteq G$, so dass
\begin{itemize}
\item $\sum_{a\in T} \weight{a} \leq \ell$ und
\item $\sum_{a\in T} \val{a} \geq w$~?
\end{itemize}
}\pause

\begin{center}\large
\redalert{Haben wir also gezeigt, dass $\Scomplclass{P}=\Scomplclass{NP}$?}
\end{center}

\end{frame}

\begin{frame}\frametitle{Pseudopolynomielle Probleme}

Unser Algorithmus zeigt nicht $\Slang{Rucksack}\in\Scomplclass{NP}$!

\begin{itemize}
\item Die Eingabe von $\Slang{Rucksack}$ hat die Länge $O(n+\log w+\log \ell)$
\item Die Zeit $O(n\ell)$ ist daher \alert{nicht} polynomiell bzgl. der Eingabelänge
\end{itemize}\pause

\defbox{Ein Problem ist in \redalert{pseudopolynomieller Zeit}, wenn es durch
eine DTM gelöst wird, die polynomiell zeitbeschränkt ist bzgl. der Eingabelänge
\alert{und} des Betrages aller Zahlen in der Eingabe.}

\emph{Äquivalent:} ein Problem ist pseudopolynomiell, wenn es in $\Scomplclass{PTime}$
liegt, wenn man alle Zahlen \alert{unär} kodiert.\medskip\pause

\examplebox{Beispiel: \Slang{Rucksack} ist pseudopolynomiell: beschränkt man das Problem auf Instanzen mit $\ell\leq p(n)$ für ein Polynom $p$, so erhält man ein Problem in \Scomplclass{P} (wobei $n$ die Zahl der Gegenstände ist).
}

\end{frame}

\begin{frame}\frametitle{Starke \Scomplclass{NP}-Vollständigkeit}

\defbox{Probleme, welche selbst dann noch \Scomplclass{NP}-vollständig sind, wenn 
man alle Zahlen unär kodiert, heißen \redalert{stark \Scomplclass{NP}-vollständig}.}\medskip\pause

\examplebox{Beispiele:
\begin{itemize}
\item \Slang{SAT} ist stark \Scomplclass{NP}-vollständig (keine Zahlen in der Eingabe)
\item \Slang{CLIQUE} ist stark \Scomplclass{NP}-vollständig
\item \ldots
\end{itemize}}\pause

\examplebox{Beispiele:
\begin{itemize}
\item \Slang{Rucksack} ist pseudopolynomiell
\item \Slang{Teilmengen-Summe} ist pseudopolynomiell
\end{itemize}}\pause

\emph{Anmerkung:} Die Reduktion $\Slang{SAT}\leq_p\Slang{Teilmengen-Summe}$ erzeugt eine polynomielle Instanz, bei der die Beträge der Zahlen exponentiell groß sind.

\end{frame}

\begin{frame}[t]\frametitle{Fake-News erkennen (1)}

Immer wieder gibt es Berichte darüber, dass irgendein neuartiges Berechnungsmodell
oder ausgefallene physikalische Anordnung $\Scomplclass{NP}$-vollständige Probleme
in polynomieller Zeit lösen könne \ldots
\smallskip\pause

$\leadsto$ Meist kann man leicht sehen, wo der Fehler liegt\bigskip

\emph{Typische Kontrollfragen:}\bigskip

\anybox{strongyellow}{Ist das Problem pseudopolynomiell?\\ Wie sind Zahlenbeträge kodiert?}\bigskip

\examplebox{Beispiel: Molekulare Anordnungen zum Lösen von \Slang{Teilmengen-Summe}, wobei
Zahlen durch eine entsprechende Zahl an Molekülen kodiert werden.}

\end{frame}

\begin{frame}[t]\frametitle{Fake-News erkennen (2)}

Immer wieder gibt es Berichte darüber, dass irgendein neuartiges Berechnungsmodell
oder ausgefallene physikalische Anordnung $\Scomplclass{NP}$-vollständige Probleme
in polynomieller Zeit lösen könne \ldots
\smallskip

$\leadsto$ Meist kann man leicht sehen, wo der Fehler liegt\bigskip

\emph{Typische Kontrollfragen:}\bigskip

\anybox{strongyellow}{Wächst der neuartige Computer mit dem Problem?\\ Ist dieses Wachstum exponentiell?}\bigskip

\examplebox{Beispiel: DNS-Rechner lösen Optimierunsaufgaben, aber müssen alle (exponentiell viele) mögliche Lösungen als DNS-Sequenz kodieren.}

\end{frame}

\begin{frame}[t]\frametitle{Fake-News erkennen (3)}

Immer wieder gibt es Berichte darüber, dass irgendein neuartiges Berechnungsmodell
oder ausgefallene physikalische Anordnung $\Scomplclass{NP}$-vollständige Probleme
in polynomieller Zeit lösen könne \ldots
\smallskip

$\leadsto$ Meist kann man leicht sehen, wo der Fehler liegt\bigskip

\emph{Typische Kontrollfragen:}\bigskip

\anybox{strongyellow}{Wird das Problem wirklich gelöst oder handelt es sich eher um ein Approximationsverfahren? Ist klar, was ein "`Berechnungsschritt"' ist und ist deren Zahl wirklich polynomiell?}\bigskip

\examplebox{Beispiel: Ein Schaltkreis mit Rückkopplungen, welcher "`in der Regel"' zu einer "`guten"' Lösung eines $\Scomplclass{NP}$-vollständigen Problems konvergiert, und das "`nach kurzer Zeit."'}

\end{frame}

\sectionSlide{NP und andere Klassen}

\begin{frame}\frametitle{\Scomplclass{P} vs. \Scomplclass{NP}}

  \alert{Bis heute ist nicht bekannt, ob $\Scomplclass{P}\neq\Scomplclass{NP}$ gilt oder nicht.}
  \begin{itemize}
  \item Intuitiv gefragt: "`Wenn es einfach ist, eine mögliche Lösung für ein Problem zu prüfen, ist es dann auch einfach, eine zu finden?"'
  \item Übertrieben: "`Can creativity be automated?"' (Wigderson, \ghost{2006)}
  \item Seit über 40 Jahren ungelöst
  \item Eines der größten offenen Probleme der Informatik und Mathematik unserer Zeit
  \item 1.000.000 USD Preisgeld für die Lösung\\("`Millenium Problem"')
  \end{itemize}

\end{frame}

\begin{frame}\frametitle{Mögliche Konsequenzen}

\alert{Falls $\Scomplclass{P}=\Scomplclass{NP}$, dann gilt}
\begin{itemize}
\item $\Scomplclass{NP}=\Scomplclass{coNP}$ (warum?)
\item und jedes nichttriviale Problem in $\Scomplclass{P}$ ist $\Scomplclass{NP}$-vollständig (warum?)
\end{itemize}
\bigskip\pause

\alert{Falls $\Scomplclass{P}\neq\Scomplclass{NP}$, dann gilt}
\begin{itemize}
\item $\Scomplclass{P}\neq\Scomplclass{coNP}$ (warum?)
\item $\Scomplclass{L}\neq\Scomplclass{NP}$ (warum?)
\item $\Scomplclass{P}\neq\Scomplclass{PSpace}$ (warum?)
\item kein $\Scomplclass{NP}$-vollständiges Problem ist in $\Scomplclass{P}$ (warum?)
\end{itemize}

\end{frame}

\begin{frame}\frametitle{Noch schwerere Fragen}

Selbst wenn $\Scomplclass{P}\neq\Scomplclass{NP}$ bewiesen werden sollte, gäbe es noch viele
offene Fragen \ldots\bigskip\pause

\anybox{strongyellow}{
\alert{$\Scomplclass{NP}\neq\Scomplclass{coNP}$?}
Falls es für die lösbaren Instanzen eines Problems kurze Zertifikate gibt,
gibt es dann immer auch kurze Zertifikate für die Unlösbarkeit von Instanzen?}%
\begin{itemize}
\item Die meisten Experten glauben das nicht
\item Falls die Antwort "`nein"' lautet, dann folgt $\Scomplclass{P}\neq\Scomplclass{NP}$
\end{itemize}\bigskip\pause

\anybox{strongyellow}{
\alert{$\Scomplclass{P}\neq\Scomplclass{NP}\cap\Scomplclass{coNP}$?}
Falls es für die Lösbarkeit und die Unlösbarkeit eines Problems kurze Zertifikate gibt,
ist das Problem dann effizient lösbar?}%
\begin{itemize}
\item Die meisten Experten glauben das nicht
\item Falls die Antwort "`nein"' lautet, dann folgt $\Scomplclass{P}\neq\Scomplclass{NP}$
\item Ist die Antwort "`ja"' und zudem $\Scomplclass{P}\neq\Scomplclass{NP}$, so folgt $\Scomplclass{NP}\neq\Scomplclass{coNP}$
\end{itemize}\bigskip

\end{frame}

\begin{frame}\frametitle{Der Satz von Ladner}

\emph{Drei mögliche Welten:}
\begin{enumerate}[(1)]
\item $\Scomplclass{P}=\Scomplclass{NP}$: jedes \Scomplclass{NP}-vollständige Problem ist in $\Scomplclass{P}$
\item $\Scomplclass{P}\neq\Scomplclass{NP}$: kein \Scomplclass{NP}-vollständiges Problem ist in $\Scomplclass{P}$
\begin{enumerate}[(2.a)]
\item Jedes \Scomplclass{NP}-Problem, welches nicht \Scomplclass{NP}-vollständig ist, liegt in \Scomplclass{P}
\item Es gibt Probleme, die nicht \Scomplclass{NP}-vollständig sind und dennoch nicht in \Scomplclass{P} liegen
\end{enumerate}
\end{enumerate}\pause\bigskip

Ladner hat gezeigt, dass Fall (2.a) unmöglich ist:

\theobox{Satz (Ladner): Falls $\Scomplclass{P}\neq\Scomplclass{NP}$, dann gibt es
Probleme in \Scomplclass{NP}, die weder \Scomplclass{NP}-vollständig sind noch in \Scomplclass{P} liegen.}

Diese Probleme heißen \redalert{\Scomplclass{NP}-intermediate} \pause{} -- wenn $\Scomplclass{P}\neq\Scomplclass{NP}$, dann sollte es viele davon geben \ldots aber wir kennen kaum Kandidaten:
\begin{itemize}
\item Faktorisierung (\Slang{Faktor-7} und ähnliche)
\item Ermitteln, ob zwei Graphen isomorph (gleich bis auf Umbenennung) sind
% \item ?
\end{itemize}


\end{frame}


\begin{frame}\frametitle{Zusammenfassung und Ausblick}

$\Slang{SAT}\leq_p\Slang{Teilmengen-Summe}\leq_p \Slang{Rucksack}$
\bigskip

Pseudopolynomielle Probleme werden einfacher, wenn man den Betrag der Zahlen in der Eingabe polynomiell beschränkt\bigskip

Offene Fragen (geordnet von "`leicht"' nach schwer):
"`$\Scomplclass{L}\neq\Scomplclass{NP}$?"',
"`$\Scomplclass{P}\neq\Scomplclass{PSpace}$?"',
"`$\Scomplclass{P}\neq\Scomplclass{NP}$?"',
"`$\Scomplclass{NP}\neq\Scomplclass{coNP}$?"'
"`$\Scomplclass{P}\neq \Scomplclass{NP}\cap\Scomplclass{coNP}$?"'
\bigskip

Es sollte eigentlich eine Menge Probleme geben, die weder in \Scomplclass{P} liegen, noch \Scomplclass{NP}-vollständig sind\bigskip

\anybox{yellow}{
Was erwartet uns als nächstes?
\begin{itemize}
\item $\Scomplclass{NL}$
\item Komplexität jenseits von $\Scomplclass{NP}$
\item Spiele
\end{itemize}
}

\end{frame}



% \begin{frame}[t]\frametitle{Literatur und Bildrechte}
% 
% \alert{Literatur}\bigskip
% 
% \begin{itemize}
% \item Richard J. Lorentz:
% \emph{Creating Difficult Instances of the Post Correspondence Problem.}
% Computers and Games 2000: 214--228
% \item John J. O'Connor, Edmund F. Robertson: \emph{Emil Leon Post.} MacTutor History of Mathematics archive, University of St Andrews. \url{http://www-history.mcs.st-andrews.ac.uk/Biographies/Post.html}
% \end{itemize}
% 
% \bigskip\bigskip
% 
% \alert{Bildrechte}\bigskip
% 
% Folie \ref{frame_post}: gemeinfrei
% 
% \end{frame}


\end{document}
