\documentclass[onlymath]{beamer}
% \documentclass[onlymath,handout]{beamer}

% Macros used by all lectures, but not necessarily by excercises

%%% General setup and dependencies:

% \usetheme[ddcfooter,nosectionnum]{tud}
\usetheme[nosectionnum,pagenum,noheader]{tud}
% \usetheme[nosectionnum,pagenum]{tud}

% Increase body font size to a sane level:
\let\origframetitle\frametitle
% \renewcommand{\frametitle}[1]{\origframetitle{#1}\normalsize}
\renewcommand{\frametitle}[1]{\origframetitle{#1}\fontsize{10pt}{13.2}\selectfont}
\setbeamerfont{itemize/enumerate subbody}{size=\small} % tud defaults to scriptsize!
\setbeamerfont{itemize/enumerate subsubbody}{size=\small}
% \setbeamerfont{normal text}{size=\small}
% \setbeamerfont{itemize body}{size=\small}

\renewcommand{\emph}[1]{\textbf{#1}}

\def\arraystretch{1.3}% Make tables even less cramped vertically

\usepackage[ngerman]{babel}
\usepackage[utf8]{inputenc}
\usepackage[T1]{fontenc}

%\usepackage{graphicx}
\usepackage[export]{adjustbox} % loads graphicx
\usepackage{import}
\usepackage{stmaryrd}
\usepackage[normalem]{ulem} % sout command
% \usepackage{times}
\usepackage{txfonts}
\usepackage{array}

% \usepackage[perpage]{footmisc} % reset footnote counter on each page -- fails with beamer (footnotes gone)
\usepackage{perpage}  % reset footnote counter on each page
\MakePerPage{footnote}

\usepackage{tikz}
\usetikzlibrary{arrows,positioning,decorations.pathreplacing}
% Inspired by http://www.texample.net/tikz/examples/hand-drawn-lines/
\usetikzlibrary{decorations.pathmorphing}
\pgfdeclaredecoration{penciline}{initial}{
    \state{initial}[width=+\pgfdecoratedinputsegmentremainingdistance,
    auto corner on length=1mm,]{
        \pgfpathcurveto%
        {% From
            \pgfqpoint{\pgfdecoratedinputsegmentremainingdistance}
                      {\pgfdecorationsegmentamplitude}
        }
        {%  Control 1
        \pgfmathrand
        \pgfpointadd{\pgfqpoint{\pgfdecoratedinputsegmentremainingdistance}{0pt}}
                    {\pgfqpoint{-\pgfdecorationsegmentaspect
                     \pgfdecoratedinputsegmentremainingdistance}%
                               {\pgfmathresult\pgfdecorationsegmentamplitude}
                    }
        }
        {%TO
        \pgfpointadd{\pgfpointdecoratedinputsegmentlast}{\pgfpoint{1pt}{1pt}}
        }
    }
    \state{final}{}
}
\tikzset{handdrawn/.style={decorate,decoration=penciline}}
\tikzset{every shadow/.style={fill=none,shadow xshift=0pt,shadow yshift=0pt}}
% \tikzset{module/.append style={top color=\col,bottom color=\col}}

% Use to make Tikz attributes with Beamer overlays
% http://tex.stackexchange.com/a/6155
\tikzset{onslide/.code args={<#1>#2}{%
  \only<#1| handout:0>{\pgfkeysalso{#2}}
}}
\tikzset{onslideprint/.code args={<#1>#2}{%
  \only<#1>{\pgfkeysalso{#2}}
}}

%%% Title -- always set this first

\newcommand{\defineTitle}[3]{
	\newcommand{\lectureindex}{#1}
	\title{Theoretische Informatik und Logik}
	\subtitle{\href{\lectureurl}{#1. Vorlesung: #2}}
	\author{\href{https://iccl.inf.tu-dresden.de/web/Markus_Kr\%C3\%B6tzsch}{Markus Kr\"{o}tzsch}\\[1ex]Lehrstuhl Wissensbasierte Systeme}
	\date{#3}
	\datecity{TU Dresden}
% 	\institute{CC-By 3.0, sofern keine anderslautenden Bildrechte angegeben sind}
}

%%% Table of contents:

\RequirePackage{ifthen}

\newcommand{\highlight}[2]{%
	\ifthenelse{\equal{#1}{\lectureindex}}{\alert{#2}}{#2}%
}

\def\myspace{-0.7ex}
\newcommand{\printtoc}{
\begin{tabular}{r@{$\quad$}l}
\highlight{1}{1.} & \highlight{1}{Willkommen/Einleitung formale Sprachen}\\[\myspace]
\highlight{2}{2.} & \highlight{2}{Grammatiken und die Chomsky-Hierarchie}\\[\myspace]
\highlight{3}{3.} & \highlight{3}{Endliche Automaten}\\[\myspace]
\highlight{4}{4.} & \highlight{4}{Complexity of FO query answering}\\[\myspace]
\highlight{5}{5.} & \highlight{5}{Conjunctive queries}\\[\myspace]
\highlight{6}{6.} & \highlight{6}{Tree-like conjunctive queries}\\[\myspace]
\highlight{7}{7.} & \highlight{7}{Query optimisation}\\[\myspace]
\highlight{8}{8.} & \highlight{8}{Conjunctive Query Optimisation / First-Order~Expressiveness}\\[\myspace]
\highlight{9}{9.} & \highlight{9}{First-Order~Expressiveness / Introduction to Datalog}\\[\myspace]
\highlight{10}{10.} & \highlight{10}{Expressive Power and Complexity of Datalog}\\[\myspace]
\highlight{11}{11.} & \highlight{11}{Optimisation and Evaluation of Datalog}\\[\myspace]
\highlight{12}{12.} & \highlight{12}{Evaluation of Datalog (2)}\\[\myspace]
\highlight{13}{13.} & \highlight{13}{Graph Databases and Path Queries}\\[\myspace]
\highlight{14}{14.} & \highlight{14}{Outlook: database theory in practice}
\end{tabular}
}

\newcommand{\overviewslide}{%
\begin{frame}\frametitle{Overview}
\printtoc
\medskip

Siehe \href{\lectureurl}{course homepage [$\Rightarrow$ link]} for more information and materials
\end{frame}
}

%%% Colours:
\usepackage{xcolor,colortbl}
\definecolor{redhighlights}{HTML}{FFAA66}
\definecolor{lightblue}{HTML}{55AAFF}
\definecolor{lightred}{HTML}{FF5522}
\definecolor{lightpurple}{HTML}{DD77BB}
\definecolor{lightgreen}{HTML}{55FF55}
\definecolor{darkred}{HTML}{CC4411}
\definecolor{darkblue}{HTML}{176FC0}%{1133AA}
\definecolor{nightblue}{HTML}{2010A0}%{1133AA}
\definecolor{alert}{HTML}{176FC0}
\definecolor{darkgreen}{HTML}{36AB14}
\definecolor{strongyellow}{HTML}{FFE219}
\definecolor{devilscss}{HTML}{666666}

\newcommand{\redalert}[1]{\textcolor{darkred}{#1}}

%%% Slide layout commands:

\newcommand{\sectionSlide}[1]{
\frame{\begin{center}
\LARGE
#1
\end{center}}
}
\newcommand{\sectionSlideNoHandout}[1]{
\frame<handout:0>{\begin{center}
\LARGE
#1
\end{center}}
}

\newcommand{\mydualbox}[3]{%
 \begin{minipage}[t]{#1}
 \begin{beamerboxesrounded}[upper=block title,lower=block body,shadow=true]%
    {\centering\usebeamerfont*{block title}#2}%
    \raggedright%
    \usebeamerfont{block body}
%     \small
    #3%
  \end{beamerboxesrounded}
  \end{minipage}
}
%
\newcommand{\myheaderbox}[2]{%
 \begin{minipage}[t]{#1}
 \begin{beamerboxesrounded}[upper=block title,lower=block title,shadow=true]%
    {\centering\usebeamerfont*{block title}\rule{0pt}{2.6ex} #2}%
  \end{beamerboxesrounded}
  \end{minipage}
}

\newcommand{\mycontentbox}[2]{%
 \begin{minipage}[t]{#1}%
 \begin{beamerboxesrounded}[upper=block body,lower=block body,shadow=true]%
    {\centering\usebeamerfont*{block body}\rule{0pt}{2.6ex}#2}%
  \end{beamerboxesrounded}
  \end{minipage}
}

\newcommand{\mylcontentbox}[2]{%
 \begin{minipage}[t]{#1}%
 \begin{beamerboxesrounded}[upper=block body,lower=block body,shadow=true]%
    {\flushleft\usebeamerfont*{block body}\rule{0pt}{2.6ex}#2}%
  \end{beamerboxesrounded}
  \end{minipage}
}

% label=180:{\rotatebox{90}{{\footnotesize\textcolor{darkgreen}{Beispiel}}}}
% \hspace{-8mm}\ghost{\raisebox{-7mm}{\rotatebox{90}{{\footnotesize\textcolor{darkgreen}{Beispiel}}}}}\hspace{8mm}
\newcommand{\examplebox}[1]{%
	\begin{tikzpicture}[decoration=penciline, decorate]
		\pgfmathsetseed{1235}
		\node (n1) [decorate,draw=darkgreen, fill=darkgreen!10,thick,align=left,text width=\linewidth, inner ysep=2mm, inner xsep=2mm] at (0,0) {#1};
% 		\node (n2) [align=left,text width=\linewidth,inner sep=0mm] at (n1.92) {{\footnotesize\raisebox{3mm}{\textcolor{darkgreen}{Beispiel}}}};
% 		\node (n2) [decorate,draw=darkgreen, fill=darkgreen!10,thick, align=left,text width=\linewidth,inner sep=2mm] at (n1.90) {{\footnotesize\raisebox{0mm}{\textcolor{darkgreen}{Beispiel}}}};
	\end{tikzpicture}%
}%

\newcommand{\codebox}[1]{%
	\begin{tikzpicture}[decoration=penciline, decorate]
		\pgfmathsetseed{1236}
		\node (n1) [decorate,draw=strongyellow, fill=strongyellow!10,thick,align=left,text width=\linewidth, inner ysep=2mm, inner xsep=2mm] at (0,0) {#1};
	\end{tikzpicture}%
}%

\newcommand{\defbox}[1]{%
	\begin{tikzpicture}[decoration=penciline, decorate]
		\pgfmathsetseed{1237}
		\node (n1) [decorate,draw=darkred, fill=darkred!10,thick,align=left,text width=\linewidth, inner ysep=2mm, inner xsep=2mm] at (0,0) {#1};
	\end{tikzpicture}%
}%

\newcommand{\theobox}[1]{%
	\begin{tikzpicture}[decoration=penciline, decorate]
		\pgfmathsetseed{1240}
		\node (n1) [decorate,draw=darkblue, fill=darkblue!10,thick,align=left,text width=\linewidth, inner ysep=2mm, inner xsep=2mm] at (0,0) {#1};
	\end{tikzpicture}%
}%

\newcommand{\anybox}[2]{%
	\begin{tikzpicture}[decoration=penciline, decorate]
		\pgfmathsetseed{1240}
		\node (n1) [decorate,draw=#1, fill=#1!10,thick,align=left,text width=\linewidth, inner ysep=2mm, inner xsep=2mm] at (0,0) {#2};
	\end{tikzpicture}%
}%


\newsavebox{\mybox}%
\newcommand{\doodlebox}[2]{%
\sbox{\mybox}{#2}%
	\begin{tikzpicture}[decoration=penciline, decorate]
		\pgfmathsetseed{1238}
		\node (n1) [decorate,draw=#1, fill=#1!10,thick,align=left,inner sep=1mm] at (0,0) {\usebox{\mybox}};
	\end{tikzpicture}%
}%


\defineTitle{8}{Beziehungen zwischen Komplexitätsklassen / Effizient lösbare Probleme}{3. Mai 2017}

\begin{document}

\maketitle

\begin{frame}\frametitle{Rückblick}

Die wichtigsten Ressourcen zur Messung von Komplexität
sind \alert{Rechenzeit} und \alert{Speicher}.
\bigskip

Die wichtigsten deterministischen Komplexitätsklassen sind:
{\footnotesize
\begin{align*}
\Scomplclass{P} = \Scomplclass{PTime} &= \bigcup_{d\geq 1} \Scomplclass{DTime}(n^d)
	& \text{\redalert{polynomielle Zeit}}
%   & \Scomplclass{NP} &= \bigcup_{k\geq 1} \Scomplclass{NTime}(n^k)
  \\[1ex]
\hspace{-1.5cm}\Scomplclass{Exp} = \Scomplclass{ExpTime} &= \bigcup_{d\geq 1} \Scomplclass{DTime}(2^{n^d})
    & \text{\redalert{exponentielle Zeit}${}^*$}
% %   & \Scomplclass{NExp} = \Scomplclass{NExpTime} &= \bigcup_{k\geq 1} \Scomplclass{NTime}(2^{n^k})
%   \\[1ex]
% \hspace{-1.5cm}\Scomplclass{2Exp} = \Scomplclass{2ExpTime} &= \bigcup_{d\geq 1} \Scomplclass{DTime}(2^{2^{n^d}})
% %   & \Scomplclass{N2Exp} = \Scomplclass{N2ExpTime} &= \bigcup_{k\geq 1} \Scomplclass{NTime}(2^{2^{n^k}})
% 	& \text{doppelt-exponentielle Zeit}
%   \\[1ex]
% \Scomplclass{E} = \Scomplclass{ETime} &= \bigcup_{d\geq 1} \Scomplclass{DTime}(2^{dn})
%    & \text{exp.\ Zeit mit linearem Exponent}
   \\[4ex]
\hspace{-1.5cm}\Scomplclass{L} = \Scomplclass{LogSpace} &= \Scomplclass{DSpace}(\log n)
%   & \Scomplclass{NL} = \Scomplclass{NLogSpace} &= \Scomplclass{NSpace}(\log n)
	& \text{\redalert{logarithmischer Speicher}}
  \\[1ex]
\Scomplclass{PSpace} &= \bigcup_{d\geq 1} \Scomplclass{DSpace}(n^d)
	& \text{\redalert{polynomieller Speicher}}
%   \\[1ex]
% \Scomplclass{ExpSpace} &= \bigcup_{d\geq 1} \Scomplclass{DSpace}(2^{n^d})
% 	& \text{exponentieller Speicher}
\end{align*}
}

\end{frame}

\begin{frame}\frametitle{Robustheit von Zeitklassen}

Zwei wichtige Erkenntnisse zur Robustheit von Zeitklassen:
\bigskip

\anybox{strongyellow}{Konstante Faktoren haben keinen Einfluss auf die Probleme, die eine zeitbeschränkte Mehrband-TM lösen kann, sofern mindestens lineare Zeit erlaubt ist (Linear Speedup Theorem).}\bigskip

\emph{Anmerkung:} Wenn nicht wenigstens lineare Zeit zur Verfügung steht, dann kann die TM nicht einmal die Eingabe lesen. Das ergibt also bei einer herkömmlichen TM wenig Sinn.
\bigskip\pause

\anybox{strongyellow}{Die Anzahl der Bänder hat lediglich einen polynomiellen (quadratischen) Einfluss auf die Probleme, die eine zeitbeschränkte TM lösen kann.}\bigskip

Das hatten wir in Formale Systeme durch Simulation mehrerer Bänder auf einem gezeigt.

\end{frame}

\begin{frame}\frametitle{Robustheit von Speicherklassen}

Bei speicherbeschränkten TMs ist die Situation sogar etwas einfacher:
\bigskip

\anybox{strongyellow}{Konstante Faktoren haben keinen Einfluss auf die Probleme, die eine speicherbeschränkte TM lösen kann.}\bigskip\pause

\emph{Beweis:} Ähnlich zu Linear Speedup, aber viel einfacher.\bigskip\pause

\anybox{strongyellow}{Die Anzahl der Bänder hat keinen Einfluss auf die Probleme, die eine speicherbeschränkte TM lösen kann.}\bigskip\pause

\emph{Beweis:} Reduktion von $k$ Bändern auf $1$ Band wie gehabt, kombiniert mit einer $1/k$ Speicherreduktion.

\end{frame}


\begin{frame}\frametitle{Beziehung von Zeit und Raum (1)}

Ist die Berechnungszeit beschränkt, so kann auch nur beschränkt viel 
Speicher genutzt werden:\bigskip

\theobox{Satz: Für jede beliebige Funktion $f:\mathbb{N}\to\mathbb{R}$ gilt
\[\Scomplclass{DTIME}(f)\subseteq \Scomplclass{DSPACE}(f).\]
}

\emph{Beweis:} Die TM benötigt immer mindestens einen Schritt um eine zusätzliche Speicherstelle zu beschreiben.\qed\bigskip

\pause Daraus folgt zum Beispiel $\Scomplclass{PTime}\subseteq\Scomplclass{PSpace}$.

\end{frame}

\begin{frame}\frametitle{Beziehung von Zeit und Raum (2)}

Andererseits ist Speicher mächtiger als Zeit, da man Speicher mehrfach verwenden kann (Zeit leider nicht):\bigskip

\theobox{Satz: Für jede beliebige Funktion $f:\mathbb{N}\to\mathbb{R}$ gilt
\[\Scomplclass{DSPACE}(f)\subseteq \Scomplclass{DTIME}(2^{O(f)}).\]}

\pause\emph{Beweis:} Gegeben sei eine Eingabe $w$ der Länge $|w|=n$.\pause
{\footnotesize
\begin{itemize}
\item Es gibt $|\Gamma|^{f(n)}=2^{\log_2(|\Gamma|)\cdot f(n)}$ Speicherbelegungen der Länge $f(n)$\pause
\item Hinzu kommen $f(n)$ mögliche Kopfpositionen und $|Q|$ \ghost{Zustände}\pause
\item Es gibt also $|Q|\cdot f(n)\cdot 2^{\log_2(|\Gamma|)\cdot f(n)}\in O(2^{O(f)})$
\ghost{TM-Konfigurationen.}\pause
\item Aus diesen kann man für eine gegebene Eingabe in polynomieller Zeit einen \alert{Konfigurationsgraph} berechnen, in dem (gerichtete) Kanten die möglichen Übergänge darstellen.\pause
\item Daraus kann man die Akzeptanz der Eingabe in polynomieller Zeit ermitteln ("`Ist von der Startkonfiguration aus eine akzeptierende Endkonfiguration erreichbar?"').
\end{itemize}}
Damit hat man das Wortproblem in Zeit $O(2^{O(f)})$ entschieden.\qed
\end{frame}

\sectionSlide{Nichtdeterministische Komplexitätsklassen}

\begin{frame}\frametitle{Ressourcen nichtdeterministischer TMs}

Bei NTMs gibt es viele mögliche Berechnungspfade.\\
$\leadsto$ Welche Pfade meinen wir, wenn wir Ressourcen beschränken?\\\pause
~\hfill{\alert{-- Alle!}}
\medskip

\defbox{%
Sei $f:\mathbb{N}\to\mathbb{R}$ eine Funktion und $\Smach{M}$ eine nichtdeterministische \ghost{TM.}
\begin{itemize}
\item $\Smach{M}$ heißt \redalert{$O(f)$-zeitbeschränkt} wenn es eine Funktion $g\in O(f)$ gibt, so
dass $\Smach{M}$ für eine beliebige Eingabe $w\in\Sigma^*$ \alert{auf jedem Berechnungspfad} nach maximal $g(|w|)$ Schritten anhält.
%
\item $\Smach{M}$ heißt \redalert{$O(f)$-speicherbeschränkt} wenn es eine Funktion $g\in O(f)$ gibt, so
dass $\Smach{M}$ für eine beliebige Eingabe $w\in\Sigma^*$ hält und zuvor \alert{auf jedem Berechnungspfad} maximal $g(|w|)$ Speicherzellen verwendet.
\end{itemize}
}

Eine zeit- oder speicherbeschränkte NTM muss also auch auf erfolglosen Pfaden ("`falsch geratene Übergänge"') garantiert
innerhalb der Ressourcengrenzen halten.

\end{frame}

\begin{frame}\frametitle{Zeit und Raum, nichtdeterministisch}

Die entsprechenden Sprachklassen werden genau wie bei deterministischen TMs definiert:\medskip

\defbox{Sei $f:\mathbb{N}\to\mathbb{R}$ eine Funktion.
\begin{itemize}
\item \redalert{\Scomplclass{NTIME}(f(n))} ist die Klasse aller Sprachen $\Slang{L}$, welche durch eine $O(f)$-zeitbeschränkte NTM entschieden werden können.
\item \redalert{\Scomplclass{NSPACE}(f(n))} ist die Klasse aller Sprachen $\Slang{L}$, welche durch eine $O(f)$-speicherbeschränkte NTM entschieden werden können.
\end{itemize}
}%\pause
% 
% \examplebox{Beispiel: $\Slang{SAT}\in\Scomplclass{DTIME}(2^n)$ und $\Slang{SAT}\in\Scomplclass{DSPACE}(n)$.}\pause
% 
% \examplebox{Beispiel: Das Halteproblem ist in keiner der Klassen $\Scomplclass{DTIME}(f(n))$ oder $\Scomplclass{DSPACE}(f(n))$. Unentscheidbare Probleme benötigen uneingeschränkten Zugang zu beliebig vielen Ressourcen einer TM.}

\end{frame}


\begin{frame}\frametitle{Nichtdeterministische Komplextitätsklassen}

Auch hier beschränken wir uns auf einige wichtige Fälle:

{\footnotesize
\begin{align*}
\Scomplclass{NP} = \Scomplclass{NPTime} &= \bigcup_{d\geq 1} \Scomplclass{NTime}(n^d)
	& \text{\redalert{nichtdet. polynomielle Zeit}}
%   & \Scomplclass{NP} &= \bigcup_{k\geq 1} \Scomplclass{NTime}(n^k)
  \\[1ex]
\hspace{-1.5cm}\Scomplclass{NExp} = \Scomplclass{NExpTime} &= \bigcup_{d\geq 1} \Scomplclass{NTime}(2^{n^d})
    & \text{\redalert{nichtdet. exponentielle Zeit}}
% %   & \Scomplclass{NExp} = \Scomplclass{NExpTime} &= \bigcup_{k\geq 1} \Scomplclass{NTime}(2^{n^k})
%   \\[1ex]
% \hspace{-1.5cm}\Scomplclass{2Exp} = \Scomplclass{2ExpTime} &= \bigcup_{d\geq 1} \Scomplclass{DTime}(2^{2^{n^d}})
% %   & \Scomplclass{N2Exp} = \Scomplclass{N2ExpTime} &= \bigcup_{k\geq 1} \Scomplclass{NTime}(2^{2^{n^k}})
% 	& \text{doppelt-exponentielle Zeit}
%   \\[1ex]
% \Scomplclass{E} = \Scomplclass{ETime} &= \bigcup_{d\geq 1} \Scomplclass{DTime}(2^{dn})
%    & \text{exp.\ Zeit mit linearem Exponent}
   \\[4ex]
\hspace{-1.5cm}\Scomplclass{NL} = \Scomplclass{NLogSpace} &= \Scomplclass{NSpace}(\log n)
%   & \Scomplclass{NL} = \Scomplclass{NLogSpace} &= \Scomplclass{NSpace}(\log n)
	& \text{\redalert{nichtdet. logarithmischer Speicher}}
  \\[1ex]
\Scomplclass{NPSpace} &= \bigcup_{d\geq 1} \Scomplclass{NSpace}(n^d)
	& \text{\redalert{nichtdet. polynomieller Speicher}}
%   \\[1ex]
% \Scomplclass{ExpSpace} &= \bigcup_{d\geq 1} \Scomplclass{DSpace}(2^{n^d})
% 	& \text{exponentieller Speicher}
\end{align*}
}\pause

\examplebox{Beispiel: Die Existenz eines Hamiltonpfads ist in $\Scomplclass{NP}$ entscheidbar. Wenn ein Hamiltonpfad existiert, dann kann er in polynomieller Zeit
erraten und überprüft werden.
}

\end{frame}

\begin{frame}\frametitle{Einfache Beobachtungen}

% Die nichtdeterministischen Klassen sind genauso robust wie die deterministischen.\bigskip

Die folgenden Konstruktionen funktionieren wie im deterministischen Fall:
\begin{itemize}
\item Zeitreduktion durch linear Speedup
\item Lineare Speicherreduktion
\item Bandreduktion von Mehrband-TMs
\end{itemize}\pause

Die Beziehungen von Zeit und Speicher bleiben ebenfalls erhalten, mit einer Besonderheit:
\medskip

\theobox{Satz: Für jede beliebige Funktion $f:\mathbb{N}\to\mathbb{R}$ gilt:
\begin{align*}
\Scomplclass{NTIME}(f) & \subseteq \Scomplclass{NSPACE}(f) \\
\Scomplclass{NSPACE}(f) &\subseteq \Scomplclass{\redalert{D}TIME}(2^{O(f)})
\end{align*}}

\pause\emph{Beweis:} Beide Fälle wie im deterministischen Fall.
Der Konfigurationsgraph ist auch hier
exponentiell groß, aber kann wie zuvor deterministisch durchsucht werden.\qed

\end{frame}

\begin{frame}\frametitle{Deterministisch vs. nichtdeterministisch}

Wir haben also nebenbei auch gezeigt: $\Scomplclass{NTIME}(f) \subseteq \Scomplclass{NSPACE}(f)\subseteq \Scomplclass{DTIME}(2^{O(f)})$.\bigskip

\theobox{Satz: Für jede beliebige Funktion $f:\mathbb{N}\to\mathbb{R}$ gilt:
\[\Scomplclass{NTIME}(f) \subseteq \Scomplclass{DTIME}(2^{O(f)})\]}

\emph{Anmerkung:} In Formale Systeme, Vorlesung 19, haben wir dieses Ergebnis durch eine alternative Konstruktion gezeigt (ausgehend von der Simulation einer beliebigen, unbeschränkten NTM durch deterministische Suche im Baum der möglichen Berechnungspfade).
\bigskip

\emph{Anmerkung 2:} Es ist bis heute nicht bekannt, ob $\Scomplclass{NTIME}(f) \subseteq \Scomplclass{DTIME}(g)$ auch für eine Funktion $g\in o(2^{O(f)})$ gilt
(Achtung: Klein-o-Notation!).
%

\end{frame}

\begin{frame}\frametitle{Was wir wissen}

Aus unseren Beobachtungen folgen verschiedene einfache Beziehungen:
\begin{itemize}
\item \alert{"`DTM $\subseteq$ NTM"':}
	$\Scomplclass{L}\subseteq \Scomplclass{NL}$,
	$\Scomplclass{P}\subseteq \Scomplclass{NP}$,
	$\Scomplclass{PSpace}\subseteq \Scomplclass{NPSpace}$,
	\ghost{$\Scomplclass{Exp}\subseteq \Scomplclass{NExp}$}
\item \alert{"`Zeit $\subseteq$ Speicher"':}
	$\Scomplclass{P}\subseteq \Scomplclass{PSpace}$,
% 	$\Scomplclass{ExpTime}\subseteq \Scomplclass{ExpSpace}$,
	$\Scomplclass{NP}\subseteq \Scomplclass{NPSpace}$
% 	$\Scomplclass{NExpTime}\subseteq \Scomplclass{NExpSpace}$,
\item \alert{"`(N)Speicher $\subseteq$ $2^{\text{(D)Zeit}}$"':}
	$\Scomplclass{NL}\subseteq \Scomplclass{P}$,
	$\Scomplclass{NPSpace}\subseteq \Scomplclass{Exp}$
\end{itemize}\bigskip\pause

Zudem besagt der berühmte \alert{Satz von Savitch}, dass speicherbeschränkte NTMs durch DTMs mit nur quadratischen Mehrkosten simuliert werden können. Daraus folgt:

\theobox{Satz (Savitch): $\Scomplclass{PSpace}= \Scomplclass{NPSpace}$.}

(ohne Beweis)\bigskip\pause

Zusammenfassung der wichtigsten bekannten Beziehungen:
\theobox{
\[\Scomplclass{L}\subseteq\Scomplclass{NL}\subseteq \Scomplclass{P}\subseteq\Scomplclass{NP}\subseteq\Scomplclass{PSpace}= \Scomplclass{NPSpace}\subseteq \Scomplclass{Exp}\subseteq \Scomplclass{NExp}\]
\vspace{-2.5ex}
}

\end{frame}

\begin{frame}\frametitle{Was wir nicht wissen}

Wir wissen:

\theobox{
\[\Scomplclass{L}\subseteq\Scomplclass{NL}\subseteq \Scomplclass{P}\subseteq\Scomplclass{NP}\subseteq\Scomplclass{PSpace}= \Scomplclass{NPSpace}\subseteq \Scomplclass{Exp}\subseteq \Scomplclass{NExp}\]
\vspace{-2.5ex}
}

\begin{itemize}
\item Wir wissen nicht, ob irgendeines dieser $\subseteq$ sogar $\subsetneq$ ist.
\item Insbesondere wissen wir nicht, ob $\Scomplclass{P}\subsetneq\Scomplclass{NP}$ oder $\Scomplclass{P}=\Scomplclass{NP}$.
\item Wir wissen nicht einmal, ob $\Scomplclass{L}\subsetneq\Scomplclass{NP}$ oder $\Scomplclass{L}=\Scomplclass{NP}$.
\end{itemize}\pause
Es wird aber vermutet, dass alle  $\subseteq$ eigentlich $\subsetneq$ sind.
% 
Bekannt ist das aber nur bei exponentiell großen Ressourcenunterschieden:

\theobox{Satz: Die folgenden Inklusionen sind echt:
\begin{itemize}
\item $\Scomplclass{NL}\subsetneq\Scomplclass{PSpace}$
\item $\Scomplclass{P}\subsetneq\Scomplclass{Exp}$
\item $\Scomplclass{NP}\subsetneq\Scomplclass{NExp}$
\end{itemize}}

(ohne Beweis; folgt aus dem sogenanntem \alert{Time (bzw. Space) Hierarchy Theorem})

\end{frame}


\sectionSlide{Effizient lösbare Probleme}

\begin{frame}\frametitle{Was bedeutet "`effizient"'?}

\emph{Intuitiv klar:} Lineare Algorithmen sind "`effizient"'
\bigskip\pause

Aber der Begriff "`linear"' ist nicht robust:
\begin{itemize}
\item Abhänging von Details des Maschinenmodells
\item Abhänging von Details der Kodierung
\end{itemize}
\bigskip
$\leadsto$ Polynomielle Zeit als robuste Verallgemeinerung von Linearzeit:\bigskip

\defbox{
\[\Scomplclass{P} = \Scomplclass{PTime} = \bigcup_{d\geq 1} \Scomplclass{DTime}(n^d)\]
}
%

\end{frame}


\begin{frame}\frametitle{Wachstum einiger Funktionen}

\begin{tikzpicture}[xscale=0.09,yscale=0.07]
	\draw[->] (0,0) -- (100,0) node[right] {$n$};
	\draw[->] (0,0) -- (0,85) node[above] {Zeit in $\mu s$};
	\node [circle,draw=none,inner sep=1pt] at (1,-4) {$1$};
	\foreach \x in {1,...,9} {
		\draw (10*\x,0.9) -- (10*\x,-0.9);
		\node [circle,draw=none,inner sep=1pt] at (10*\x,-4) {$\x0$};
	}
	\foreach \y in {0,...,8} {
		\draw (-0.9,10*\y) -- (0.9,10*\y);
		\pgfmathtruncatemacro\yexp{\y*3}
		\node [circle,draw=none,inner sep=1pt] at (-5,10*\y) {$10^{\yexp}$};
	}
	
	\draw[dashed,-,darkblue] (0,25.9) -- (100,25.9);
	\node [draw=none,align=right,darkblue] at (95,28) {\footnotesize 1 Min.};
	\draw[dashed,-,darkblue] (0,36.4) -- (100,36.4);
	\node [draw=none,align=right,darkblue] at (95,38.4) {\footnotesize 1 Tag};
	\draw[dashed,-,darkblue] (0,44.95) -- (100,44.95);
	\node [draw=none,align=right,darkblue] at (95,47) {\footnotesize 1 Jahr};
	\draw[dashed,-,darkblue] (0,54.96) -- (100,54.96);
	\node [draw=none,align=left,darkblue] at (91,57) {\footnotesize 1000 Jahre};
	\draw[dashed,-,darkblue] (0,78.76) -- (100,78.76);
	\node [draw=none,align=right,darkblue] at (50,81) {\footnotesize Alter des Universums};

	\visible<2->{\draw[domain=1:96,smooth,variable=\x,darkred,line width=0.3mm] plot ({\x},{1.003433*10*log2(\x)}) node [right] {$n^{10}$};}
	\draw[domain=1:96,smooth,variable=\x,darkred,line width=0.3mm] plot ({\x},{1.003433*3*log2(\x)}) node [right] {$n^3$};
	\draw[domain=1:96,smooth,variable=\x,darkred,line width=0.3mm] plot ({\x},{1.003433*2*log2(\x)}) node [right] {$n^2$};
	\draw[domain=1:96,smooth,variable=\x,darkred,line width=0.3mm] plot ({\x},{1.003433*log2(\x)}) node [right] {$n$};
	\draw[domain=1:86,smooth,variable=\x,darkred,line width=0.3mm] plot ({\x},{1.003433*\x-1}) node [right] {$2^n$};
	
	\draw[-,darkred,line width=0.3mm] (1,0) -- (2,1) -- (3,2.59) -- (4,4.6) -- (5,6.93) -- (6,9.52) --
		(7,12.34) -- (8,15.35) -- (9,18.53) -- (10,21.87) -- (11,25.34) -- (12,28.93) --
		(13,32.65) -- (14,36.47) -- (15,40.39) -- (16,44.40) -- (17,48.50) --
		(18,52.69) -- (19,56.95) -- (20,61.29) -- (21,65.69) -- (22,70.17) --
		(23,74.71) -- (24,79.31) -- (25,83.97) node [above] {$n!$};
\end{tikzpicture}

\end{frame}

\begin{frame}\frametitle{Polynomiell = effizient?}

Wir verwenden \Scomplclass{P} als mathematisches Modell für die Klasse
der praktisch lösbaren Probleme.
\bigskip

\emph{Aber:} diese Übereinstimmung ist nicht perfekt
\begin{itemize}
\item Polynome hohen Grades können sehr schnell wachsen
\item Die konstanten Faktoren können auch sehr groß sein (und Linear Speedup hilft in der Praxis wenig)
\end{itemize}\bigskip\pause

\emph{Dennoch:} \Scomplclass{P} ist von praktischem wie theoretischem Interesse
\begin{itemize}
\item \alert{Praxis:} Die meisten polynomiellen Probleme erlauben Algorithmen in $O(x^2)$ oder $O(x^3)$, während man zum Beispiel $O(x^{10})$ selten antrifft
\item \alert{Theorie:} Unabhängig von der tatsächlichen Laufzeit liefert uns \Scomplclass{P} tiefe Einsichten in die Struktur eines Problems
\end{itemize}

\end{frame}

\begin{frame}\frametitle{Probleme in \Scomplclass{P}}

Aus der Vorlesung Formale Systeme kennen wir bereits ein
typisches \Scomplclass{P}-Problem.\bigskip

\emph{Rückblick:}
\begin{itemize}
\item Eine \alert{Hornklausel} ist eine aussagenlogische Formel der Form $p_1\wedge\ldots\wedge p_n\to q$ mit $n\geq 0$
{\tiny (bei $n=0$ ergibt sich einfach $q$ -- ein \alert{Fakt})}
\item Eine Formel ist \alert{erfüllbar} wenn sie für eine Wertezuweisung auf wahr abgebildet wird
\item Eine Menge von Formeln ist erfüllbar, wenn es eine Wert-\\zuweisung gibt, die alle
ihre Elemente gleichzeitig wahr macht
\end{itemize}\bigskip

\defbox{\redalert{Erfüllbarkeit propositionaler Horn-Formeln}
(\Slang{HornSAT}) ist das folgende Entscheidungsproblem:\\[1ex]
\emph{Gegeben:} Eine Menge aussagenlogischer Formeln Hornklauseln\\
\emph{Frage:} Ist diese Menge erfüllbar?
}

\end{frame}

\begin{frame}\frametitle{\Slang{2SAT}}

Ein weiteres Beispiel für ein polynomiell lösbares Problem aus der Aussagenlogik:
\bigskip

\emph{Rückblick:}
\begin{itemize}
\item Ein \alert{Literal} ist ein aussagenlogisches Atom oder ein negiertes aussagenlogisches Atom
\item Eine \alert{Klausel} ist eine Disjunktion von Literalen, die man oft einfach als Menge darstellt
\item Eine Formel ist in \alert{Klauselform} ist eine Konjunktion von Klauseln,
ebenfalls dargestellt als Menge
\end{itemize}\pause

\defbox{\Slang{2SAT} ist das folgende Entscheidungsproblem:\\[1ex]
\emph{Gegeben:} Eine aussagenlogische Formel $F$ in Klauselform, bei der jede Klausel höchstens zwei Literale enthält\\
\emph{Frage:} Ist $F$ erfüllbar?
}

\end{frame}

\begin{frame}\frametitle{\Slang{2SAT} ist in \Scomplclass{P}}

\theobox{Satz: $\Slang{2SAT}\in\Scomplclass{P}$}

\pause\emph{Beweis:} Dazu wollen wir einen polynomiellen Algorithmus angeben.
\pause\bigskip

Der Resolutionsalgorithmus aus der Vorlesung Formale Systeme erfüllt den Zweck ohne jegliche Abwandlungen:
\begin{itemize}
\item Ein Resolutionsschritt kombiniert zwei Klauseln $\{p,L_1\}$ und $\{\neg p,L_2\}$
zu einer neuen Klausel $\{L_1,L_2\}$
\item Aus Zweier-Klauseln entstehen also immer wieder Klauseln mit höchstens zwei Literalen
\item Es gibt nur quadratisch viele Klauseln mit höchstens zwei Literalen
\item Der Algorithmus terminiert also nach polynomieller Zeit\qed
\end{itemize}

\end{frame}

% \sectionSlide{Noch effizienter}

\begin{frame}\frametitle{Effizienter als \Scomplclass{P}?}

Wenn wir eine robuste Klasse wollen, die Linearzeit-Algorithmen enthält, dann enthält sie auch beliebige polynomiellen Algorithmen.\bigskip

\alert{Gibt es Probleme, die noch einfacher sind?}
\smallskip\pause

\begin{itemize}
\item \emph{Sub-lineare Zeit} funktioniert mit dem normalen TM-Modell nicht, da man
in dieser Zeit nicht einmal die Eingabe lesen kann {\tiny (erfordert Rechenmodelle mit einer Form von Parallelverarbeitung \ldots)}
\item \emph{Sub-linearer Speicher} ist machbar, wenn man ein getrenntes schreibgeschütztes Eingabeband erlaubt (siehe letzte Vorlesung)
\end{itemize}
$\leadsto$ Komplexitätsklassen \Scomplclass{L} und \Scomplclass{NL}

\end{frame}

\begin{frame}\frametitle{Was kann \Scomplclass{L}?}

\emph{Intuition:} ein Algorithmus mit logarithmischem Speicher kann:
\begin{itemize}
\item Eine feste Anzahl an binärkodierten Zählern speichern, die nicht größer als $O(n)$ werden
\item Eine feste Anzahl an "`Pointern"' auf eine Position der Eingabe speichern
\item Den Inhalt von Zählern und Speicherstellen miteinander vergleichen
\end{itemize}

Damit kann man bereits viele einfache Algorithmen umsetzen\pause

\examplebox{Beispiel: Die Sprache aller Wörter über $\{\Sterm{0},\Sterm{1}\}$, welche
die gleiche Anzahl der Symbole $\Sterm{0}$ und $\Sterm{1}$ enthalten, ist in \Scomplclass{L}:
\begin{itemize}
\item Wir verwenden zwei Zähler für die beiden Zahlen
\item Die TM liest das Wort von links nach rechts und erhöht jeweils den entsprechenden Zähler
\item Am Ende wird der Wert beider Zähler verglichen
\end{itemize}}

\end{frame}

\begin{frame}\frametitle{Noch ein Beispiel in \Scomplclass{L}}

\examplebox{Beispiel: Die Sprache \Slang{Palindrom}, welche Wörter enthält, die von hinten gelesen genauso lauten wie von vorn, ist in \Scomplclass{L}:
\begin{itemize}
\item Wir verwenden zwei Pointer in die Eingabe, einer auf die erste und einer auf die letzte Eingabezelle
\item Die TM vergleicht die Speicherinhalte bei den Pointern und verschiebt sie anschließend um eine Zelle in Richtung Wortmitte
\item Das Wort wird akzeptiert, wenn die Pointer sich treffen und alle Vergleiche erfolgreich waren
\end{itemize}

}

\end{frame}

\sectionSlide{Reduktionen}

\begin{frame}\frametitle{Effizient berechenbare Funktionen}

Bisher haben wir Entscheidungsprobleme betrachtet.
Man kann unsere Definitionen leicht auf andere Berechnungsprobleme übertragen.\medskip

\defbox{Eine totale Funktion $f$ ist in \redalert{polynomieller Zeit berechenbar}, wenn es eine polynomiell zeitbeschränkte DTM $\Smach{M}$ gibt, die $f$ berechnet, d.h. mit $f=f_{\Smach{M}}$.
}

\pause Bei logarithmischen Speicherschranken müssen wir vorsichtig sein: Das Ergebnis könnte größer sein als der Arbeitsspeicher!

\defbox{Eine totale Funktion $f$ ist in \redalert{logarithmischen Speicher berechenbar}, wenn es eine 3-Band DTM $\Smach{M}$ gibt, die $f$ wie folgt berechnet:
(1) die Eingabe befindet sich auf dem schreibgeschützten \alert{Eingabeband}, (2) die DTM verwendet maximal $O(\log n)$ Speicherzellen auf dem \alert{Arbeitsband},
(3) die Ausgabe wird auf das \alert{Ausgabeband} geschrieben (pro Zelle einmaliges Schreiben, kein Lesen).%
% \begin{itemize}
% \item Die Eingabe befindet sich auf dem schreibgeschützten Eingabeband
% \item Die DTM verwendet maximal $O(\log n)$ Speicherzellen auf dem Arbeitsband
% \item Die Ausgabe wird auf das Ausgabeband geschrieben (pro Zelle einmaliges Schreiben, kein Lesen)
% \end{itemize}
}

\end{frame}

\begin{frame}\frametitle{Polynomielle Many-One-Reduktionen}

Mithilfe effizient berechenbarer Funktionen können wir ein Problem
effizient auf ein anderes reduzieren.\bigskip

\defbox{Eine polynomiell berechenbare Funktion $f:\Sigma^*\to\Sigma^*$ ist eine \redalert{polynomielle Many-One-Reduktion} von einer Sprache
\Slang{P} auf eine Sprache \Slang{Q} (in Symbolen: $\Slang{P}\leq_p \Slang{Q}$), wenn für alle Wörter $w\in\Sigma^*$ gilt:\\[1ex]
\narrowcentering{$w\in\Slang{P}\quad$ genau dann wenn $\quad f(w)\in\Slang{Q}$}}\pause\bigskip

Wir sprechen oft einfach von \redalert{polynomiellen Reduktionen}.
\bigskip

Logarithmische Many-One-Reduktionen könnten analog definiert werden (wir werden uns damit nicht näher beschäftigen.

\end{frame}

\begin{frame}\frametitle{Komplexität durch Reduktion zeigen}

\emph{Idee:} Wenn man ein Problem $\Slang{A}$ leicht auf ein leichtes Problem $\Slang{B}$ reduzieren kann,
dann ist $\Slang{A}$ ebenfalls leicht.\bigskip

\theobox{Satz: Falls $\Slang{A}\leq_p\Slang{B}$ und $\Slang{B}\in\Scomplclass{PTime}$, dann ist $\Slang{A}\in\Scomplclass{PTime}$.}

\emph{Beweis:} Folgt durch Hintereinanderausführung der polynomiellen Prozeduren, da die Summe und Komposition von Polynomen ein Polynom ist.\qed
\bigskip\pause

Die Umkehrung wird uns später noch interessieren:

\theobox{Satz: Falls $\Slang{A}\leq_p\Slang{B}$ und $\Slang{A}\notin\Scomplclass{PTime}$, dann ist $\Slang{B}\notin\Scomplclass{PTime}$.}\bigskip

Dazu mehr in den kommenden Vorlesungen \ldots

\end{frame}

\begin{frame}\frametitle{Beispiel}

% Als Beispiel betrachten wir folgendes Problem:

\defbox{\redalert{2-Färbbarkeit} (\Slang{2Col}) ist das folgende Problem:\\
\emph{Gegeben:} Ein ungerichteter Graph $G$\\
\emph{Frage:} Kann man die Knoten von $G$ mit zwei Farben (rot und blau) so einfärben, dass keine gleichfarbigen Knoten durch Kanten verbunden sind?}

Man kann $\Slang{2Col}\in\Scomplclass{P}$ leicht durch Reduktion auf \Slang{2SAT} zeigen:
\begin{itemize}
\item Für jeden Knoten $v$ führen wir ein Atom $p_v$ ein
\item Für jede Kante $v\to w$ führen wir zwei Klauseln ein: $\{p_v,p_w\}$ und $\{\neg p_v,\neg p_w\}$\pause
\end{itemize}
Dies kann man offenbar in polynomieller Zeit berechnen.\bigskip\pause% (sogar in logarithmischer!).\bigskip

Die Reduktionseigenschaft gilt:
\begin{itemize}
\item Wenn der Graph 2-färbbar ist, dann verwenden wir die Farben als Wahrheitswerte und erhalten eine
	erfüllende Zuweisung
\item Wenn die Formel erfüllbar ist, dann erhalten wir umgekehrt aus der Wertzuweisung eine korrekte 2-Färbung.\qed
\end{itemize}

\end{frame}


\begin{frame}\frametitle{Zusammenfassung und Ausblick}

Die grundlegenden Beziehungen der Komplexitätsklassen sind
\[\Scomplclass{L}\subseteq\Scomplclass{NL}\subseteq \Scomplclass{P}\subseteq\Scomplclass{NP}\subseteq\Scomplclass{PSpace}= \Scomplclass{NPSpace}\subseteq \Scomplclass{Exp}\subseteq \Scomplclass{NExp}\]

Die Klassen \Scomplclass{P}, \Scomplclass{L} und \Scomplclass{NL} sind mathematische Modelle für effiziente Algorithmen\bigskip

Mit polynomiellen Reduktionen kann man aus der Komplexität eines Problems auf die eines anderen schließen\bigskip

\anybox{yellow}{
Was erwartet uns als nächstes?
\begin{itemize}
\item Das erste Repetitorium (nächster Termin)
\item $\Scomplclass{NP}$
\item $\Scomplclass{NL}$
\end{itemize}
}

\end{frame}

% \begin{frame}[t]\frametitle{Literatur und Bildrechte}
% 
% \alert{Literatur}\bigskip
% 
% \begin{itemize}
% \item Richard J. Lorentz:
% \emph{Creating Difficult Instances of the Post Correspondence Problem.}
% Computers and Games 2000: 214--228
% \item John J. O'Connor, Edmund F. Robertson: \emph{Emil Leon Post.} MacTutor History of Mathematics archive, University of St Andrews. \url{http://www-history.mcs.st-andrews.ac.uk/Biographies/Post.html}
% \end{itemize}
% 
% \bigskip\bigskip
% 
% \alert{Bildrechte}\bigskip
% 
% Folie \ref{frame_post}: gemeinfrei
% 
% \end{frame}


\end{document}