\documentclass[onlymath]{beamer}
% \documentclass[onlymath,handout]{beamer}

% Macros used by all lectures, but not necessarily by excercises

%%% General setup and dependencies:

% \usetheme[ddcfooter,nosectionnum]{tud}
\usetheme[nosectionnum,pagenum,noheader]{tud}
% \usetheme[nosectionnum,pagenum]{tud}

% Increase body font size to a sane level:
\let\origframetitle\frametitle
% \renewcommand{\frametitle}[1]{\origframetitle{#1}\normalsize}
\renewcommand{\frametitle}[1]{\origframetitle{#1}\fontsize{10pt}{13.2}\selectfont}
\setbeamerfont{itemize/enumerate subbody}{size=\small} % tud defaults to scriptsize!
\setbeamerfont{itemize/enumerate subsubbody}{size=\small}
% \setbeamerfont{normal text}{size=\small}
% \setbeamerfont{itemize body}{size=\small}

\renewcommand{\emph}[1]{\textbf{#1}}

\def\arraystretch{1.3}% Make tables even less cramped vertically

\usepackage[ngerman]{babel}
\usepackage[utf8]{inputenc}
\usepackage[T1]{fontenc}

%\usepackage{graphicx}
\usepackage[export]{adjustbox} % loads graphicx
\usepackage{import}
\usepackage{stmaryrd}
\usepackage[normalem]{ulem} % sout command
% \usepackage{times}
\usepackage{txfonts}
\usepackage{array}

% \usepackage[perpage]{footmisc} % reset footnote counter on each page -- fails with beamer (footnotes gone)
\usepackage{perpage}  % reset footnote counter on each page
\MakePerPage{footnote}

\usepackage{tikz}
\usetikzlibrary{arrows,positioning,decorations.pathreplacing}
% Inspired by http://www.texample.net/tikz/examples/hand-drawn-lines/
\usetikzlibrary{decorations.pathmorphing}
\pgfdeclaredecoration{penciline}{initial}{
    \state{initial}[width=+\pgfdecoratedinputsegmentremainingdistance,
    auto corner on length=1mm,]{
        \pgfpathcurveto%
        {% From
            \pgfqpoint{\pgfdecoratedinputsegmentremainingdistance}
                      {\pgfdecorationsegmentamplitude}
        }
        {%  Control 1
        \pgfmathrand
        \pgfpointadd{\pgfqpoint{\pgfdecoratedinputsegmentremainingdistance}{0pt}}
                    {\pgfqpoint{-\pgfdecorationsegmentaspect
                     \pgfdecoratedinputsegmentremainingdistance}%
                               {\pgfmathresult\pgfdecorationsegmentamplitude}
                    }
        }
        {%TO
        \pgfpointadd{\pgfpointdecoratedinputsegmentlast}{\pgfpoint{1pt}{1pt}}
        }
    }
    \state{final}{}
}
\tikzset{handdrawn/.style={decorate,decoration=penciline}}
\tikzset{every shadow/.style={fill=none,shadow xshift=0pt,shadow yshift=0pt}}
% \tikzset{module/.append style={top color=\col,bottom color=\col}}

% Use to make Tikz attributes with Beamer overlays
% http://tex.stackexchange.com/a/6155
\tikzset{onslide/.code args={<#1>#2}{%
  \only<#1| handout:0>{\pgfkeysalso{#2}}
}}
\tikzset{onslideprint/.code args={<#1>#2}{%
  \only<#1>{\pgfkeysalso{#2}}
}}

%%% Title -- always set this first

\newcommand{\defineTitle}[3]{
	\newcommand{\lectureindex}{#1}
	\title{Theoretische Informatik und Logik}
	\subtitle{\href{\lectureurl}{#1. Vorlesung: #2}}
	\author{\href{https://iccl.inf.tu-dresden.de/web/Markus_Kr\%C3\%B6tzsch}{Markus Kr\"{o}tzsch}\\[1ex]Lehrstuhl Wissensbasierte Systeme}
	\date{#3}
	\datecity{TU Dresden}
% 	\institute{CC-By 3.0, sofern keine anderslautenden Bildrechte angegeben sind}
}

%%% Table of contents:

\RequirePackage{ifthen}

\newcommand{\highlight}[2]{%
	\ifthenelse{\equal{#1}{\lectureindex}}{\alert{#2}}{#2}%
}

\def\myspace{-0.7ex}
\newcommand{\printtoc}{
\begin{tabular}{r@{$\quad$}l}
\highlight{1}{1.} & \highlight{1}{Willkommen/Einleitung formale Sprachen}\\[\myspace]
\highlight{2}{2.} & \highlight{2}{Grammatiken und die Chomsky-Hierarchie}\\[\myspace]
\highlight{3}{3.} & \highlight{3}{Endliche Automaten}\\[\myspace]
\highlight{4}{4.} & \highlight{4}{Complexity of FO query answering}\\[\myspace]
\highlight{5}{5.} & \highlight{5}{Conjunctive queries}\\[\myspace]
\highlight{6}{6.} & \highlight{6}{Tree-like conjunctive queries}\\[\myspace]
\highlight{7}{7.} & \highlight{7}{Query optimisation}\\[\myspace]
\highlight{8}{8.} & \highlight{8}{Conjunctive Query Optimisation / First-Order~Expressiveness}\\[\myspace]
\highlight{9}{9.} & \highlight{9}{First-Order~Expressiveness / Introduction to Datalog}\\[\myspace]
\highlight{10}{10.} & \highlight{10}{Expressive Power and Complexity of Datalog}\\[\myspace]
\highlight{11}{11.} & \highlight{11}{Optimisation and Evaluation of Datalog}\\[\myspace]
\highlight{12}{12.} & \highlight{12}{Evaluation of Datalog (2)}\\[\myspace]
\highlight{13}{13.} & \highlight{13}{Graph Databases and Path Queries}\\[\myspace]
\highlight{14}{14.} & \highlight{14}{Outlook: database theory in practice}
\end{tabular}
}

\newcommand{\overviewslide}{%
\begin{frame}\frametitle{Overview}
\printtoc
\medskip

Siehe \href{\lectureurl}{course homepage [$\Rightarrow$ link]} for more information and materials
\end{frame}
}

%%% Colours:
\usepackage{xcolor,colortbl}
\definecolor{redhighlights}{HTML}{FFAA66}
\definecolor{lightblue}{HTML}{55AAFF}
\definecolor{lightred}{HTML}{FF5522}
\definecolor{lightpurple}{HTML}{DD77BB}
\definecolor{lightgreen}{HTML}{55FF55}
\definecolor{darkred}{HTML}{CC4411}
\definecolor{darkblue}{HTML}{176FC0}%{1133AA}
\definecolor{nightblue}{HTML}{2010A0}%{1133AA}
\definecolor{alert}{HTML}{176FC0}
\definecolor{darkgreen}{HTML}{36AB14}
\definecolor{strongyellow}{HTML}{FFE219}
\definecolor{devilscss}{HTML}{666666}

\newcommand{\redalert}[1]{\textcolor{darkred}{#1}}

%%% Slide layout commands:

\newcommand{\sectionSlide}[1]{
\frame{\begin{center}
\LARGE
#1
\end{center}}
}
\newcommand{\sectionSlideNoHandout}[1]{
\frame<handout:0>{\begin{center}
\LARGE
#1
\end{center}}
}

\newcommand{\mydualbox}[3]{%
 \begin{minipage}[t]{#1}
 \begin{beamerboxesrounded}[upper=block title,lower=block body,shadow=true]%
    {\centering\usebeamerfont*{block title}#2}%
    \raggedright%
    \usebeamerfont{block body}
%     \small
    #3%
  \end{beamerboxesrounded}
  \end{minipage}
}
%
\newcommand{\myheaderbox}[2]{%
 \begin{minipage}[t]{#1}
 \begin{beamerboxesrounded}[upper=block title,lower=block title,shadow=true]%
    {\centering\usebeamerfont*{block title}\rule{0pt}{2.6ex} #2}%
  \end{beamerboxesrounded}
  \end{minipage}
}

\newcommand{\mycontentbox}[2]{%
 \begin{minipage}[t]{#1}%
 \begin{beamerboxesrounded}[upper=block body,lower=block body,shadow=true]%
    {\centering\usebeamerfont*{block body}\rule{0pt}{2.6ex}#2}%
  \end{beamerboxesrounded}
  \end{minipage}
}

\newcommand{\mylcontentbox}[2]{%
 \begin{minipage}[t]{#1}%
 \begin{beamerboxesrounded}[upper=block body,lower=block body,shadow=true]%
    {\flushleft\usebeamerfont*{block body}\rule{0pt}{2.6ex}#2}%
  \end{beamerboxesrounded}
  \end{minipage}
}

% label=180:{\rotatebox{90}{{\footnotesize\textcolor{darkgreen}{Beispiel}}}}
% \hspace{-8mm}\ghost{\raisebox{-7mm}{\rotatebox{90}{{\footnotesize\textcolor{darkgreen}{Beispiel}}}}}\hspace{8mm}
\newcommand{\examplebox}[1]{%
	\begin{tikzpicture}[decoration=penciline, decorate]
		\pgfmathsetseed{1235}
		\node (n1) [decorate,draw=darkgreen, fill=darkgreen!10,thick,align=left,text width=\linewidth, inner ysep=2mm, inner xsep=2mm] at (0,0) {#1};
% 		\node (n2) [align=left,text width=\linewidth,inner sep=0mm] at (n1.92) {{\footnotesize\raisebox{3mm}{\textcolor{darkgreen}{Beispiel}}}};
% 		\node (n2) [decorate,draw=darkgreen, fill=darkgreen!10,thick, align=left,text width=\linewidth,inner sep=2mm] at (n1.90) {{\footnotesize\raisebox{0mm}{\textcolor{darkgreen}{Beispiel}}}};
	\end{tikzpicture}%
}%

\newcommand{\codebox}[1]{%
	\begin{tikzpicture}[decoration=penciline, decorate]
		\pgfmathsetseed{1236}
		\node (n1) [decorate,draw=strongyellow, fill=strongyellow!10,thick,align=left,text width=\linewidth, inner ysep=2mm, inner xsep=2mm] at (0,0) {#1};
	\end{tikzpicture}%
}%

\newcommand{\defbox}[1]{%
	\begin{tikzpicture}[decoration=penciline, decorate]
		\pgfmathsetseed{1237}
		\node (n1) [decorate,draw=darkred, fill=darkred!10,thick,align=left,text width=\linewidth, inner ysep=2mm, inner xsep=2mm] at (0,0) {#1};
	\end{tikzpicture}%
}%

\newcommand{\theobox}[1]{%
	\begin{tikzpicture}[decoration=penciline, decorate]
		\pgfmathsetseed{1240}
		\node (n1) [decorate,draw=darkblue, fill=darkblue!10,thick,align=left,text width=\linewidth, inner ysep=2mm, inner xsep=2mm] at (0,0) {#1};
	\end{tikzpicture}%
}%

\newcommand{\anybox}[2]{%
	\begin{tikzpicture}[decoration=penciline, decorate]
		\pgfmathsetseed{1240}
		\node (n1) [decorate,draw=#1, fill=#1!10,thick,align=left,text width=\linewidth, inner ysep=2mm, inner xsep=2mm] at (0,0) {#2};
	\end{tikzpicture}%
}%


\newsavebox{\mybox}%
\newcommand{\doodlebox}[2]{%
\sbox{\mybox}{#2}%
	\begin{tikzpicture}[decoration=penciline, decorate]
		\pgfmathsetseed{1238}
		\node (n1) [decorate,draw=#1, fill=#1!10,thick,align=left,inner sep=1mm] at (0,0) {\usebox{\mybox}};
	\end{tikzpicture}%
}%


\defineTitle{6}{Unentscheidbare Probleme formaler Sprachen}{26. April 2017}

\begin{document}

\maketitle

\begin{frame}\frametitle{Rückblick}

Zwei wesentliche Erkenntnisse der letzten Vorlesung:

\begin{itemize}
\item Praktisch alle interessanten Fragen zu Turingmaschinen sind unentscheibar (Rice)
\item Es gibt unentscheidbare Probleme, die nicht direkt mit Berechnung zu tun haben (Post)
\end{itemize}

\end{frame}

\begin{frame}[t]\frametitle{Nachtrag: Von PCP zu MPCP (1)}

Es fehlt noch eine Reduktion von MPCP auf PCP.\bigskip

\theobox{Satz: Es gibt eine Many-One-Reduktion vom 
modifizierten PCP auf PCP.}

\pause\emph{Beweis:} Wir verwenden zwei zusätzliche Symbole $\Sterm{\#}$ und 
$\Sterm{\blacksquare}$. Für ein Wort $w=a_1\cdots a_\ell$
definieren wir:
\begin{align*}
{}_{\Sterm{\#}}w_{\Sterm{\#}} &= \Sterm{\#}{a_1}\Sterm{\#}\cdots\Sterm{\#}{a_\ell}\Sterm{\#} &
w_{\Sterm{\#}} &= {a_1}\Sterm{\#}\cdots\Sterm{\#}{a_\ell}\Sterm{\#} &
{}_{\Sterm{\#}}w &= \Sterm{\#}{a_1}\Sterm{\#}\cdots\Sterm{\#}{a_\ell}
\end{align*}
\pause Die gesuchte Reduktion bildet jetzt ein MPCP\\[1ex]
\narrowcentering{$\left[\begin{matrix}x_1\\y_1\end{matrix}\right]\quad\ldots\quad\left[\begin{matrix}x_k\\y_k\end{matrix}\right]$}\\[1ex]
ab auf das PCP\\[1ex]
\narrowcentering{$%
\left[\begin{matrix}{}_{\Sterm{\#}}x_1{}_{\Sterm{\#}}\\{}_{\Sterm{\#}}y_1\end{matrix}\right]
\quad\left[\begin{matrix}x_1{}_{\Sterm{\#}}\\{}_{\Sterm{\#}}y_1\end{matrix}\right]
\quad\ldots
\quad\left[\begin{matrix}x_k{}_{\Sterm{\#}}\\{}_{\Sterm{\#}}y_k\end{matrix}\right]
\quad\left[\begin{matrix}\Sterm{\blacksquare}\\\Sterm{\#\blacksquare}\end{matrix}\right]
$}\\[1ex]


\end{frame}

\begin{frame}[t]\frametitle{Nachtrag: Von PCP zu MPCP (2)}

% Es fehlt noch eine Reduktion von MPCP auf PCP.\bigskip
% 
% \theobox{Satz: Es gibt eine Many-One-Reduktion vom 
% modifizierten PCP auf PCP.}

\emph{Beweis (Fortsetzung):} Wir erhalten also das folgende PCP\\[1ex]
\narrowcentering{$%
\left[\begin{matrix}{}_{\Sterm{\#}}x_1{}_{\Sterm{\#}}\\{}_{\Sterm{\#}}y_1\end{matrix}\right]
\quad\left[\begin{matrix}x_1{}_{\Sterm{\#}}\\{}_{\Sterm{\#}}y_1\end{matrix}\right]
\quad\ldots
\quad\left[\begin{matrix}x_k{}_{\Sterm{\#}}\\{}_{\Sterm{\#}}y_k\end{matrix}\right]
\quad\left[\begin{matrix}\Sterm{\blacksquare}\\\Sterm{\#\blacksquare}\end{matrix}\right]
$}\\[1ex]
%
Es ist nicht schwer zu zeigen, dass dies genau dann eine Lösung hat, wenn das ursprüngliche MPCP eine hat:\pause
\begin{itemize}
\item "`$\Leftarrow$"' Wenn das MPCP eine Lösung hat, dann erhalten wir leicht eine entsprechende Lösung für das PCP, wobei jedes Symbol zusätzlich von $\Sterm{\#}$ umgeben ist und das Wort auf $\Sterm{\blacksquare}$ endet\pause
\item "`$\Rightarrow$"' Wenn das PCP eine Lösung hat, dann muss es mit dem ersten Wortpaar beginnen, da nur dieses Wortpaar gleiche Anfangssymbole hat. Durch Weglassen aller $\Sterm{\#}$ und 
$\Sterm{\blacksquare}$ entsteht wieder eine Lösung des MPCP.\qed
\end{itemize}

\end{frame}


\sectionSlide{Unentscheidbare Probleme formaler Sprachen}


\begin{frame}\frametitle{Wiederholung (Vorlesung Formale Systeme)}

Wir schreiben $\Slang{L}(G)$ für die Sprache, welche durch die Grammatik $G$ erzeugt wird.\bigskip

\theobox{Satz (aus Formale Systeme): Das \alert{Schnittproblem regulärer Grammatiken} ist entscheidbar.\\
\emph{Gegeben:} Reguläre Grammatiken $G_1$ und $G_2$\\
\emph{Frage:} Ist $\Slang{L}(G_1)\cap\Slang{L}(G_2)\neq\emptyset$?
}
\pause\emph{Beweisskizze:} Für reguläre Grammatiken $G_1$ und $G_2$ kann man $\Slang{L}(G_1)\cap\Slang{L}(G_2)$ durch einen Automaten darstellen (Produkt-\\konstruktion). Automaten kann man leicht auf Leerheit testen.\qed\bigskip\pause

\theobox{Satz (aus Formale Systeme): Das \alert{Schnittproblem kontextfreier Grammatiken} ist unentscheidbar.\\
\emph{Gegeben:} Kontextfreie Grammatiken $G_1$ und $G_2$\\
\emph{Frage:} Ist $\Slang{L}(G_1)\cap\Slang{L}(G_2)\neq\emptyset$?
}

\end{frame}

\begin{frame}\frametitle{CFG-Schnittproblem unentscheidbar (1)}

\theobox{Satz: Das \alert{Schnittproblem kontextfreier Grammatiken} ist unentscheidbar.\\
\emph{Gegeben:} Kontextfreie Grammatiken $G_1$ und $G_2$\\
\emph{Frage:} Ist $\Slang{L}(G_1)\cap\Slang{L}(G_2)\neq\emptyset$?
}

\pause\emph{Beweis:} Durch Many-One-Reduktion vom PCP:\pause\medskip

\begin{itemize}
\item Für eine gegebene PCP-Instanz $P$
\item konstruieren wir kontextfreie Grammatiken $G_x$ und $G_y$,\\[1ex]
	so dass gilt:
\item $P$ hat eine Lösung genau dann wenn $\Slang{L}(G_x)\cap\Slang{L}(G_y)\neq\emptyset$.
\end{itemize}

\end{frame}

\begin{frame}[t]\frametitle{CFG-Schnittproblem unentscheidbar (2)}

\emph{Beweis:} Sei $\left[\begin{matrix}x_1\\y_1\end{matrix}\right]\quad\ldots\quad\left[\begin{matrix}x_k\\y_k\end{matrix}\right]$ eine PCP-Instanz mit Alphabet $\Sigma$.
\medskip\pause

Die Grammatik $G_x$ wird definiert als $\tuple{V,\Sigma_k,P_x,\Snterm{S}}$ mit
\begin{itemize}
\item $V=\{\Snterm{S}\}$
\item $\Sigma_k = \Sigma \cup\{1,\ldots,k\}$ (o.b.d.A. sei dies eine dijunkte Vereinigung)
\item $P_x$ ist die Menge aller Regeln 
\begin{align*}
\Snterm{S}\to i\Snterm{S}x_i\quad\text{ und }\quad\Snterm{S}\to i x_i && \text{für alle $1\leq i\leq k$}
\end{align*}

\end{itemize}
Damit ist $G_x$ leicht berechenbar.\\
$G_y=\tuple{V,\Sigma_k,P_y,\Snterm{S}}$ wird analog definiert.\bigskip\pause

Damit ergibt sich:
\begin{itemize}
\item $\Slang{L}(G_x)=\{ i_\ell\cdots i_1 x_{i_1}\cdots x_{i_\ell}\mid \ell\geq 1\text{ und } i_1,\ldots,i_\ell\in\{1,\ldots,k\}\}$ und
\item $\Slang{L}(G_y)=\{ i_\ell\cdots i_1 y_{i_1}\cdots y_{i_\ell}\mid \ell\geq 1\text{ und } i_1,\ldots,i_\ell\in\{1,\ldots,k\}\}$
\end{itemize}

\end{frame}

\begin{frame}[t]\frametitle{CFG-Schnittproblem unentscheidbar (3)}

\emph{Beweis:} Wie soeben erkannt:
\begin{itemize}
\item $\Slang{L}(G_x)=\{ i_\ell\cdots i_1 x_{i_1}\cdots x_{i_\ell}\mid \ell\geq 1\text{ und } i_1,\ldots,i_\ell\in\{1,\ldots,k\}\}$ und
\item $\Slang{L}(G_y)=\{ i_\ell\cdots i_1 y_{i_1}\cdots y_{i_\ell}\mid \ell\geq 1\text{ und } i_1,\ldots,i_\ell\in\{1,\ldots,k\}\}$
\end{itemize}\bigskip\pause

Damit folgt:\medskip

\begin{tabular}{rl}
& $\Slang{L}(G_x)\cap\Slang{L}(G_y)\neq\emptyset$\\\pause
gdw. & es gibt eine Sequenz $i_1,\ldots,i_\ell\in\{1,\ldots,k\}$ mit $\ell\geq 1$, so dass:\\
	& $ i_\ell\cdots i_1 x_{i_1}\cdots x_{i_\ell} = i_\ell\cdots i_1 y_{i_1}\cdots y_{i_\ell}$\\\pause
gdw. & es gibt eine Sequenz $i_1,\ldots,i_\ell\in\{1,\ldots,k\}$ mit $\ell\geq 1$, so dass:\\
	& $ x_{i_1}\cdots x_{i_\ell} = y_{i_1}\cdots y_{i_\ell}$\\\pause
gdw. &  Die PCP-Instanz $P$ hat eine Lösung\qed\\
\end{tabular}

\end{frame}


\begin{frame}\frametitle{Wiederholung (Vorlesung Formale Systeme)}

\emph{Wir wissen:}
\begin{itemize}
\item Kontextfreien Grammatiken kann man als Kellerautomaten darstellen und umgekehrt\\
 (diese Umformung ist berechenbar)
\item Deterministische kontextfreie Sprachen kann man als deterministische Kellerautomaten darstellen
\end{itemize}
\bigskip

\theobox{Satz (Formale Systeme):
\begin{itemize}
\item Das Leerheitsproblem für kontextfreie Grammatiken ist entscheidbar
\item Kontextfreie Sprachen sind unter Vereinigung abgeschlossen
\item Deterministische kontextfreie Sprachen sind unter Komplement abgeschlossen
\end{itemize}}

\end{frame}

\begin{frame}\frametitle{Eine einfache Beobachtung}

Die Grammatiken $G_x$ und $G_y$ aus dem vorigen Beweis kann man leicht
als deterministische Kellerautomaten darstellen\pause:
\begin{itemize}
\item Die Indices $i_\ell\cdots i_1$ lassen sich deterministisch einlesen und auf dem Stack ablegen
\item Sobald der Wortteil $x_{i_1}\cdots x_{i_\ell}$ beginnt, wird der Stack abgearbeitet und jeweils nur das Wort für den aktuellen Index akzeptiert
\end{itemize}
\bigskip\pause

Wir haben also auch schon gezeigt:

\theobox{Korollar: Das \alert{Schnittproblem deterministischer Kellerautomaten} ist unentscheidbar.\\
\emph{Gegeben:} Deterministische Kellerautomaten $\Smach{M}_1$ und $\Smach{M}_2$\\
\emph{Frage:} Ist $\Slang{L}(\Smach{M}_1)\cap\Slang{L}(\Smach{M}_2)\neq\emptyset$?
}

\end{frame}

\begin{frame}[t]\frametitle{CFG-Äquivalenz (1)}

\theobox{Satz: Das \alert{Äquivalenzproblem kontextfreier Grammatiken} ist unentscheidbar.\\
\emph{Gegeben:} Kontextfreie Grammatiken $G_1$ und $G_2$\\
\emph{Frage:} Ist $\Slang{L}(G_1)=\Slang{L}(G_2)$?
}

\emph{Beweis:} Durch Many-One-Reduktion vom Komplement des Schnittproblems.\pause

\begin{itemize}
\item Wir verwenden $G_x$ und $G_y$ aus dem vorigen Beweis\pause
\item Wir wissen, wie man einen det.\ Kellerautomaten $\Smach{M}_x$ für $G_x$ konstruiert\pause
\item $\Smach{M}_x$ kann man komplementieren: sei $\overline{\Smach{M}}_x$ der Automat für
die Sprache $\overline{\Slang{L}(\Smach{M}_x)}$\pause
\item Für $\overline{\Smach{M}}_x$ kann man eine Grammatik berechnen: sei $\overline{G}_x$ die Grammatik für die Sprache $\Slang{L}(\overline{\Smach{M}}_x)$\pause
\item Kontextfreie Grammatiken kann man vereinigen: sei $G_{\overline{x}y}$ die Grammatik mit $\Slang{L}(G_{\overline{x}y})= \Slang{L}(\overline{G}_x)\cup\Slang{L}(G_y)$ 
\end{itemize}

\end{frame}

\begin{frame}[t]\frametitle{CFG-Äquivalenz (2)}

\theobox{Satz: Das \alert{Äquivalenzproblem kontextfreier Grammatiken} ist unentscheidbar.\\
\emph{Gegeben:} Kontextfreie Grammatiken $G_1$ und $G_2$\\
\emph{Frage:} Ist $\Slang{L}(G_1)=\Slang{L}(G_2)$?
}

\emph{Beweis:} Wir behaupten: \["`\Slang{L}(G_x)\cap\Slang{L}(G_y)\stackrel{?}{=}\emptyset"' \mapsto "`\Slang{L}(G_{\overline{x}y})\stackrel{?}{=}\Slang{L}(\overline{G}_x)"'\] ist die gesuchte Reduktion.\bigskip\pause

\begin{tabular}{rrl}
$\Slang{L}(G_x)\cap\Slang{L}(G_y)=\emptyset$ \pause
	& gdw. & $\Slang{L}(G_y)\subseteq\Slang{L}(\overline{G}_x)$\\\pause
	& gdw. & $\Slang{L}(G_y)\cup\Slang{L}(\overline{G}_x)=\Slang{L}(\overline{G}_x)$\\\pause
	& gdw. & $\Slang{L}(G_{\overline{x}y})=\Slang{L}(\overline{G}_x)$\\[1ex]
\end{tabular}

Die Behauptung folgt, da das Komplement des Schnittproblems unentscheidbar ist.\qed

\end{frame}

\begin{frame}\frametitle{Diskussion}

\emph{Anmerkung 1:} $G_{\overline{x}y}$ ist nicht unbedingt deterministisch. Der Beweis
gilt also nicht für deterministische CFGs. In der Tat ist Äquivalenz dort (mit viel Aufwand) entscheidbar.\\[1ex]
{\tiny (S\'{e}nizergues: L(A)=L(B)? decidability results from complete formal systems, 2001; der komplexe Beweis zeigt Semi-Entscheidbarkeit des Problems und seines Komplements, also keine Zeitgrenzen!)

}
\bigskip\pause

\emph{Anmerkung 2:} Aus der Unentscheidbarkeit der CFG-Äquivalenz folgt -- durch einfache Many-One-Reduktion -- die Unentscheidbarkeit der Äquivalenz aller Formalismen, in die man CFGs leicht übersetzen kann:
\begin{itemize}
\item Kellerautomaten
\item kontextsensitive Grammatiken/LBAs
\item LOOP-Programme
\item Typ-0-Grammatiken/Turingmaschinen/WHILE-Programme
\item \ldots
\end{itemize}

\end{frame}

\begin{frame}\frametitle{Unentscheidbare Probleme für Typ 1}

Wir halten noch einmal fest:

\theobox{Satz: Für kontextsensitive Grammatiken $G_1$ und $G_2$ sind
die folgenden Fragen unentscheidbar:
\begin{enumerate}[(1)]
\item Äquivalenz: $\Slang{L}(G_1)=\Slang{L}(G_2)$?
\item Schnitt: $\Slang{L}(G_1)\cap\Slang{L}(G_2)=\emptyset$?
\item Leerheit: $\Slang{L}(G_1)=\emptyset$?
\end{enumerate}}

\emph{Beweis:} (1) und (2) gelten, weil alle kontextfreien Grammatiken auch
kontextsensitive Grammatiken sind (offensichtliche Many-One-Reduktion).
\bigskip

(3) gilt, da kontextsensitive Grammatiken unter Schnitten abgeschlossen sind (siehe
Vorlesung Formale Systeme), so dass man Schnitt auf Leerheit reduzieren kann.\qed

\end{frame}

\begin{frame}\frametitle{Semi-Entscheidbarkeit}

\emph{Beobachtung 1:} Das Schnittproblem ist semi-entscheidbar: zähle alle
Wörter von $\Slang{L}(G_1)$ auf und teste jeweils, ob sie in $\Slang{L}(G_2)$
liegen.
\bigskip\pause

\emph{Beobachtung 2:} Das Komplement des Schnittproblems ist demnach nicht semi-entscheidbar.
Ebenso ist also das Äquivalenzproblem nicht semi-entscheidbar (wegen Many-One-Reduktion).
\bigskip\pause

\emph{Beobachtung 3:} Das Komplement des Äquivalenzproblems ist semi-entscheidbar: zähle abwechselnd Wörter von $\Slang{L}(G_1)$ und $\Slang{L}(G_2)$ auf und teste jeweils, ob sie nicht in $\Slang{L}(G_2)$ bzw. $\Slang{L}(G_1)$ liegen.

\end{frame}

\sectionSlide{Unentscheidbarkeiten}

\begin{frame}\frametitle{Das schwerste unentscheidbare Problem?}

Wir haben gesehen (Übung):\smallskip

\theobox{Satz: Jedes semi-entscheidbare Problem kann auf das Halteproblem
many-one-reduziert werden.}\medskip\pause

Demnach kann man außerdem Komplemente semi-entscheidbarer Probleme ("`co-semi-entscheidbare"' Probleme) auf das Halteproblem Turing-reduzieren.\bigskip

Mit anderen Worten: Wenn man das Halteproblem lösen könnte, dann könnte man
jedes (co-)semi-entscheidbare Problem lösen.\pause

\begin{center}
\alert{Ist das Halteproblem das schwerste unentscheidbare Problem?}\\
{\tiny (Sind alle unentscheidbaren Probleme auf das Halteproblem Turing-reduzierbar?)}
\end{center}

\end{frame}

\begin{frame}\frametitle{Das schwerste unentscheidbare Problem?}

\begin{center}
\alert{Ist das Halteproblem das schwerste unentscheidbare Problem?}\\
{\tiny (Sind alle unentscheidbaren Probleme auf das Halteproblem Turing-reduzierbar?)}
\end{center}

\bigskip
\pause \narrowcentering{{\Large\redalert{Nein, sicher nicht.}}}
\bigskip\pause

\emph{Beweisskizze:} Wir können uns Turing-Reduktionen als TMs vorstellen, die Subroutinen
aufrufen dürfen.

\begin{itemize}
\item Selbst ohne die Details der formalen Definition ist klar: Solche TMs müssen weiterhin
endlich beschreibbar sein.
\item Daher gibt es nur abzählbar viele solcher TMs.
\item Es gibt aber überabzählbar viele Probleme.
\end{itemize}

Also sind die meisten Probleme nicht durch Turing-Reduktionen auf das Halteproblem
lösbar.\qed

\end{frame}

\begin{frame}\frametitle{Noch unentscheidbarere Probleme}

Gibt es auch konkrete unentscheidbare Probleme, die nicht
mithilfe von $\Slang{P}_{\text{halt}}$ lösbar sind?
\bigskip\pause

Ja, zum Beispiel folgendes:

\defbox{Wir betrachten folgendes Problem $\Slang{P}^2_{\text{halt}}$:\\
\emph{Gegeben:} ein Wort $w$ und eine DTM $\Smach{M}$, welche $\Slang{P}_{\text{halt}}$ als Subroutine verwenden darf\\
\emph{Frage:} Hält $\Smach{M}$ auf $w$?
}

Dies ist sozusagen ein Haltproblem höherer Ordnung.\bigskip\pause

Ein noch schwereres Problem $\Slang{P}^3_{\text{halt}}$ ist
das Halteproblem für TMs, die $\Slang{P}^2_{\text{halt}}$ als Subroutine
verwenden dürfen\\
$\leadsto$ \alert{eine unendliche Hierarchie unentscheidbarer Probleme}
\bigskip\pause

Und selbst all diese Probleme sind nur abzählbar viele \ldots

\end{frame}

\begin{frame}\frametitle{Das leichteste unentscheidbare Problem?}

\begin{center}
\alert{Ist das Halteproblem das leichteste unentscheidbare Problem?}\\
{\tiny (Ist das Halteproblem auf alle unentscheidbaren Probleme Turing-reduzierbar?)}
\end{center}

\bigskip
\pause \narrowcentering{{\Large\redalert{Nein, auch das gilt nicht.}}}
\bigskip

Die Situation ist ziemlich kompliziert:
\begin{itemize}
\item Es gibt unentscheibare Probleme $\Slang{A}$ und $\Slang{B}$, so dass
\item $\Slang{A}\leq_T \Slang{P}_{\text{halt}}$ und $\Slang{B}\leq_T \Slang{P}_{\text{halt}}$, aber
\item $\Slang{A}\not\leq_T \Slang{B}$ und $\Slang{B}\not\leq_T \Slang{A}$
\end{itemize}

Man kann also mit $\not\leq_T$ nicht einmal alle Klassen unentscheidbarer Probleme
in eine totale Ordnung bringen.

{\tiny Bewiesen in 1956 (unabhängig!) von Friedberg (USA) und Muchnik (USSR)}
\bigskip

Allerdings sind diese Probleme sehr künstlich.

\end{frame}

\begin{frame}\frametitle{Wozu das alles?}

Die Untersuchung der Struktur des Unentscheidbaren hat
sehr viele Fragen betrachtet und beantwortet.\\
$\leadsto$ Forschungsgegenstand der \alert{Berechenbarkeitstheorie}
\bigskip\pause

Offensichtliche Frage: \redalert{Bringt uns das praktische Einsichten?}
\bigskip\pause

"`Jain"':
\begin{itemize}
\item Einerseits sind alle unentscheidbaren Probleme praktisch unlösbar
\item Andererseits kann der Grad der Unentscheidbarkeit ein Hinweis auf die Schwere
entscheidbarer Teilprobleme sein
\end{itemize}

\end{frame}

\begin{frame}\frametitle{Beispiel: Noch ein Problem}

\defbox{Das \redalert{Universalitätsproblem von TMs} fragt, ob eine TM alle Eingaben
akzeptiert:\\
\emph{Gegeben:} Turingmaschine $\Smach{M}$ über Eingabealphabet $\Sigma$\\
\emph{Frage:} Ist $\Slang{L}(\Smach{M})=\Sigma^*$?}

\pause Das Universalitätsproblem vom TMs ist schwerer als das Halteproblem (aber Turing-reduzierbar auf $\Slang{P}^2_{\text{halt}}$). Das zeigt sich auch bei Sonderfällen:
\begin{itemize}
\item \alert{Kontextfreie Grammatiken:} Wortproblem und Leerheitsproblem entscheidbar; Universalität unentscheidbar
\item \alert{Endliche Automaten:} Wortproblem und Leerheitsproblem effizient lösbar (polynomiell); Universalität PSpace-hart (nur exponentielle Algorithmen bekannt)
\end{itemize}

\end{frame}

\sectionSlide{Euklid als Informatiker}

\begin{frame}\frametitle{Geometrie nach Euklid}

Etwa im 3. Jhd. v. Chr. veröffentlicht Euklid sein Lehrbuch 
\redalert{Die~Elemente} und begründet darin die euklidische Geometrie.
\bigskip

Zentrales Thema der euklidischen Geometrie ist die
Konstruktion mit den \redalert{euklidischen Werkzeugen:}

\begin{itemize}
\item \alert{Lineal:} beliebig lang, aber ohne Markierungen
\item \alert{Zirkel:} zeichnet Kreise, aber trägt bei Euklid keine Längen ab (kollabierend)
\end{itemize}

Die Konstruktion mit diesen idealen Werkzeugen gilt bei den Griechen und noch
Jahrhunderte später Königsdisziplin der Mathematik

\end{frame}

\begin{frame}\frametitle{Konstruktion mit Zirkel und Lineal}


\alert{Erlaubte Konstruktionsschritte:}
\begin{enumerate}[(1)]
\item Ziehen einer beliebig langen Geraden durch zwei verschiedene Punkte
\item Zeichnen eines Kreises mit einen gegebenen Mittelpunkt, der durch einen gegebenen Punkt verläuft
\item Abtragen einer Strecke mit dem Zirkel\\
{\tiny Bei Euklid nicht direkt erlaubt, aber Euklid selbst hat bewiesen, dass diese Operation als Makro mithilfe der Operationen (1) und (2) darstellbar ist
}
\end{enumerate}


\end{frame}

\begin{frame}\frametitle{Beispiel}

Man kann ein Quadrat wie folgt konstruieren:\bigskip

\narrowcentering{%
\begin{tikzpicture}[scale=0.4]
% \path[use as bounding box] (-3.2,0) rectangle (3.5,-4); % add "draw" to see it
% \pgfmathsetseed{6571}
% \draw[help lines] (0,0) grid (5,5);

\draw[line width=0.3mm] (-4,0) -- (10,0);
\draw[line width=0.3mm] (1,-0.1) -- (1,0.1);
\draw[line width=0.3mm] (5,-0.1) -- (5,0.1);

\visible<2-7>{\draw[line width=1mm,darkred] (1,0) -- (5,0);}

\visible<3-10>{\draw (5,0) circle (4cm);}

\visible<4-7>{\draw (1,-8) arc (270:430:8cm);}
\visible<5-7>{\draw (9,8) arc (90:250:8cm);}

\visible<6-10>{\draw[line width=0.3mm] (5,7.5) -- (5,-8);}

\visible<7->{\draw[line width=1mm,darkred] (1,0) -- (5,0) -- (5,4);}

\visible<9-10>{\draw (1,0) circle (4cm);
\draw (-3,-8) arc (270:430:8cm);
\draw (5,8) arc (90:250:8cm);
\draw[line width=0.3mm] (1,8) -- (1,-7.5);}

\visible<10-11>{\draw[line width=1mm,darkred] (1,0) -- (5,0) -- (5,4) -- (1,4) -- cycle;}
\end{tikzpicture}}

\end{frame}

\begin{frame}\frametitle{Weitere Konstruktionsbeispiele}

Es lassen sich zahlreiche weitere Konstruktionen durchführen, z.B.:
\begin{itemize}
\item Halbierung eines Winkels
\item Konstruktion des regelmäßigen Sechsecks
\item Konstruktion eines flächengleichen Quadrates aus einem gegebenen Rechteck\pause
\item Konstruktion des regelmäßigen 17-Ecks (entdeckt von Gauss -- "`Durch angestrengtes Nachdenken \ldots am Morgen \ldots (ehe ich aus dem Bette aufgestanden war)"')\pause
\item Konstruktion des regelmäßigen 65537-Ecks (Hermes)
\item \ldots
\end{itemize}


\end{frame}

\begin{frame}\frametitle{Rechnen mit Euklid}

1637: Ren\'{e} Descartes publiziert die Idee des Koordinatensystems\smallskip

$\leadsto$ Geometrie wird numerisch!
\bigskip\pause

\emph{Beispiel:} Beginnend mit Punkten an den Koordinaten $(0,0)$ und $(1,0)$
können wir einen neuen Punkt konstruieren:\bigskip

\narrowcentering{%
\begin{tikzpicture}[scale=0.4]
% \path[use as bounding box] (-3.2,0) rectangle (3.5,-4); % add "draw" to see it
% \pgfmathsetseed{6571}
% \draw[help lines] (0,0) grid (5,5);

\draw[line width=0.3mm] (-4,0) -- (10,0);
\draw[line width=0.3mm] (1,-0.1) -- (1,0.1);
\draw[line width=0.3mm] (5,-0.1) -- (5,0.1);

\node [circle,draw=none,inner sep=1pt] at (0.95,-0.7) {0};
\node [circle,draw=none,inner sep=1pt] at (4.95,-0.7) {1};

\draw[line width=0.5mm] (1,0) -- (5,0) -- (5,4) -- (1,4) -- cycle;

\draw[darkred,line width=0.5mm] (6.65685424949238019521,0) arc (0:45:5.65685424949238019521cm);

\visible<2>{\node [circle,draw=none,inner sep=1pt, darkred] at (6.6,-0.7) {?};}
\visible<3>{\node [circle,draw=none,inner sep=1pt, darkred] at (6.6,-0.7) {$\sqrt{2}$};}

\end{tikzpicture}}\bigskip

\visible<3>{Wir haben also $\sqrt{2}$ "`berechnet"'!}

\end{frame}

\begin{frame}\frametitle{Was kann dieser "`Rechner"'?}
\pause

Man kann Geometrie durch Gleichungen darstellen:
\begin{itemize}
\item Gerade durch Punkte $(a,b)$ und $(c,d)$:\\ $y=\frac{d-b}{c-a}x + \frac{bc-da}{c-a}$
\item Kreis um Mittelpunkt $(a,b)$ durch Punkt $(c,d)$: $(x-a)^2 + (y-b)^2 = (a-c)^2 + (b-d)^2$
\end{itemize}

Zeichnen: Systeme solcher Gleichungen grafisch Lösen
\bigskip\pause

\alert{Es stellt sich heraus:} Alle so konstruierbaren Zahlen ergeben sich mit folgenden Rechnungen:
\begin{itemize}
\item Addition und Subtraktion
\item Multiplikation und Division
\item Ziehen der Quadratwurzel
\end{itemize}\pause

$\leadsto$ Unmöglich ("`euklidisch unberechenbar"') sind zum Beispiel die Konstruktion von $\pi$ (\alert{"`Quadratur des Kreises"'})
und die Berechnung von Kubikwurzeln (\alert{"`Verdoppelung des Würfels"'})

\end{frame}

\begin{frame}\frametitle{Euklid statt Turing?}

Liefert uns das eine alternatives Berechenbarkeitsmodell?\pause
\bigskip

\alert{Vermutlich nicht:}
\begin{itemize}
\item Exaktes Zeichnen ist nicht physisch implementierbar (es gibt z.B. keinen perfekten Kreis)
\item Die Ergebnisse sind nicht exakt ablesbar (Messfehler)
\end{itemize}\pause

Dennoch illustriert das wichtige Ideen der Informatik:\bigskip

\anybox{strongyellow}{\vspace{-5mm}\begin{center}
Informatik erforscht, was Computer sind und welche Probleme man mit ihnen lösen kann.\end{center}}\bigskip

Man sollte trotz Church-Turing immer neu fragen, was Rechnen noch sein kann \ldots

\end{frame}


\begin{frame}\frametitle{Zusammenfassung und Ausblick}

Die Unentscheidbarkeit vieler Probleme der Sprachtheorie lässt sich gut durch Reduktion vom Postschen Korrespondenzproblem zeigen\bigskip

Es gibt mehr als eine Art von Unentscheidbarkeit\bigskip

Euklid hätte vielleicht auch Informatiker sein können\bigskip

\anybox{yellow}{
Was erwartet uns als nächstes?
\begin{itemize}
\item Methoden, zur Unterteilung entscheidbarer Probleme: Komplexität
\item Effizienz von Turingmaschinen
\item Praktisch lösbare Probleme
\end{itemize}
}

\end{frame}

% \begin{frame}[t]\frametitle{Literatur und Bildrechte}
% 
% \alert{Literatur}\bigskip
% 
% \begin{itemize}
% \item Richard J. Lorentz:
% \emph{Creating Difficult Instances of the Post Correspondence Problem.}
% Computers and Games 2000: 214--228
% \item John J. O'Connor, Edmund F. Robertson: \emph{Emil Leon Post.} MacTutor History of Mathematics archive, University of St Andrews. \url{http://www-history.mcs.st-andrews.ac.uk/Biographies/Post.html}
% \end{itemize}
% 
% \bigskip\bigskip
% 
% \alert{Bildrechte}\bigskip
% 
% Folie \ref{frame_post}: gemeinfrei
% 
% \end{frame}


\end{document}